%!TEX root = ../../main.tex
\subsection{Internet of things (IoT) Big Data Anomaly Detection}

IoT is identified as a network of  devices that is interconnected with softwares, servers, sensors and etc. In the field of Internet of things (IoT), data generated by weather stations, RFID tags, IT infrastructure
components, and some other sensors are mostly time series sequential data. Anomaly detection in these IoT networks identifies fraudulent, faulty behaviour of these massive scale of interconnected devices. The challenges associated  in outlier detection is that heterogeneous devices are interconnected which renders the system more complex. A thorough overview on using  deep learning (DL), to facilitate the analytics and learning in the IoT domain is presented by ~\cite{mohammadi2018deep}. In this section we present some of the deep anomaly detection techniques used in this domain in Table ~\ref{tab:iotBigDataAnomalyDetect}.

%%%%%%% Begin table iot BigData Anomaly Detect
\begin{table*}
\begin{center}
\caption{Examples of Deep learning anomaly detection Techniques Used in Internet of things (IoT) Big Data Anomaly Detection.
        \\ AE: Autoencoders, LSTM : Long Short Term Memory Networks
        \\ DBN : Deep Belief Networks.}
  \label{tab:iotBigDataAnomalyDetect}
    \begin{tabular}{ | l | p{2cm} | p{6cm} |}
    \hline
     \textbf{Techniques}  & \textbf{Section} & \textbf{References} \\ \hline
     AE & Section ~\ref{sec:ae} & ~\cite{luo2018distributed},~\cite{mohammadi2018neural} \\\hline
     DBN & Section ~\ref{sec:dnn} & ~\cite{kakanakova2017outlier} \\ \hline
     LSTM & Section ~\ref{sec:rnn_lstm_gru} & ~\cite{zhang2018lstm},~\cite{mudassar2018unsupervised}\\ \hline
    \end{tabular}
\end{center}
\end{table*}
%%%%%%%%% End of iot BigData Anomaly Detect





