% !TEX root=../main.tex
We propose a one-class neural network (OC-NN) model to detect anomalies in complex data sets. OC-NN
combines the ability of deep networks to extract progressively rich representation of data
with the one-class objective of creating a tight envelope around normal  data. The OC-NN approach
breaks new ground for the following crucial reason:  data representation in the hidden layer is
driven by the OC-NN objective and is thus customized for anomaly detection. This is a departure
from other approaches which use a hybrid approach of learning deep features using an autoencoder
and then feeding the features into a separate anomaly detection method like one-class SVM (OC-SVM).
The hybrid OC-SVM approach is sub-optimal because it is unable to influence representational
learning in the hidden layers. A comprehensive set of experiments demonstrate that on complex data
sets (like CIFAR and GTSRB), OC-NN performs on par with state-of-the-art methods and outperformed conventional shallow methods in some scenarios.

\keywords {one class svm,  anomalies detection, outlier detection, deep learning }
