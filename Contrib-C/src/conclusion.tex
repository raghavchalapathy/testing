% !TEX root=../main.tex
\section{Conclusion}
\label{sec:ocnn_conclusion}
In this paper, we have proposed a one-class neural network (OC-NN) approach for anomaly detection.
OC-NN uses a one-class SVM (OC-SVM) like loss function to train a neural network.
The advantage of OC-NN is that the features of the hidden layers are constructed for the specific
task of anomaly detection. This approach is substantially different from recently proposed
hybrid approaches which use deep learning features as input into an anomaly detector.  Feature
extraction in hybrid approaches is generic and not aware of the anomaly detection task. To learn
the parameters of the OC-NN network we have proposed a novel alternating minimization approach and
have shown that the optimization of a subproblem in OC-NN is equivalent to a quantile selection
% problem. Experiments on complex image and sequential data sets demonstrates that OC-NN is
% highly accurate. For future work, we would like to build and deploy an end to end system for anomaly
% detection based on OC-NN.


