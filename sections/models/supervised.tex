%!TEX root = ../../main.tex
\subsection{Supervised Deep Anomaly Detection}
\label{sec:supervisedDAD}
~\cite{gornitz2013toward} argue that supervised anomaly detection techniques are superior in performance compared to  unsupervised learning techniques since these techniques leverage labeled samples and depend on discriminating data classes. Supervised anomaly detection illustrated in Section~\ref{supervised_learning} is used to learn a model (classifier) from a set of labeled data instances (training) and then, classify a test instance into one of the classes using the learnt model (testing).\\
\textbf{Assumptions :} \\
Supervised learning methods depend on separating data classes whereas unsupervised
techniques focus on explaining and understanding the characteristics of data. A supervised classifier based on the labels available for training phase, can be grouped into two broad categories: multi-class and semi-supervised discussed in Section ~\ref{semi_supervised_DAD}. Multi-class classification based anomaly detection techniques assume that the training data contains labeled instances of  multiple normal classes ~\cite{shilton2013combined,jumutc2014multi,kim2015deep,erfani2017shared}. Multi-class anomaly detection techniques learn a classifier to distinguish between each normal class against the rest of the classes. In general, supervised deep learning-based classification schemes for anomaly detection have two subnetworks, a feature extraction network followed by a classifier network. Deep models  requires the availability of multiple classes for training and extremely large number of training samples (in the order of thousands or millions) to effectively learn feature representations to discriminate various class instances. Due to, lack of availability of clean data labels supervised deep anomaly detection techniques are not so popular as semi-supervised and unsupervised methods.

\textbf{Computational Complexity :} \\
The computational complexity of supervised deep anomaly detection methods based techniques depends on the dimensionality of the input data. High dimensional data tend to have more hidden layers to ensure meaningfull hierarchical learning of input features. As the number of hidden layers of deep learning networks increases, the computational complexity also increases accordingly, which increases the model training and update time.


\textbf{Advantages and Disadvantages of Classification Based Techniques :}\\
The advantages of supervised deep anomaly detection techniques are as follows:
\begin{itemize}
\item Supervised deep anomaly detection techniques, especially the multi-class techniques, can make
use of powerful algorithms that can distinguish between instances belonging to
different classes.\\
\item The testing phase of classification based techniques is fast since each test instance
needs to be compared against the pre-computed model.
\end{itemize}
The disadvantages of classification based techniques are as follows:
\begin{itemize}
\item  Multi-class supervised techniques require accurate labels for various normal classes, which is often not available.
\item Supervised techniques fail to separate normal from anomalous data , if the feature space is highly complex and non-linear.
\end{itemize}












