%!TEX root = ../../main.tex
\subsection{Medical Anomaly Detection:}
\label{sec:medical_anomaly_detection}

 A survey of theoretical and practical aspects of deep learning in medical and bioinformatics is described by ~\cite{min2017deep,cao2018deep,zhao2016deep,khan2018review}. Finding rare objects and events (anomalies) in areas such as medical image analysis, Clinical electroencephalography (EEG) records which consists of vast amounts of complex data enables to diagnose and provide preventive treatments for a variety of medical conditions. Deep learning based architectures are employed with great success as illustrated in Table ~\ref{tab:medicalanomalyDetect}. The vast amount of imbalanced data in medical domain presents significant challenges anomaly detection. Additionally deep learning techniques for long have been considered as black-box techniques, i,e even though deep learning models produce outstanding performance, these models lack interpretability. In recent times these works by ~\cite{gugulothusparse,amarasinghe2018toward,choi2018doctor} demonstrate models with good interpretability and performance.
%%%%%%% Begin table fraud detection
\begin{table*}
\begin{center}
  \caption{Examples of Deep learning anomaly detection Techniques Used for medical anomaly detection.
          \\AE: Autoencoders, LSTM : Long Short Term Memory Networks
          \\GRU: Gated Recurrent Unit, RNN: Recurrent Neural Networks
          \\CNN: Convolutional Neural Networks,VAE: Variational Autoencoders
          \\GAN: Generative Adversarial Networks, KNN: K-nearest neighbours
          \\RBM: Restricted Boltzmann Machines.}
  \label{tab:medicalanomalyDetect}
    \begin{tabular}{ | l | p{4cm} | p{7cm} |}
    \hline
    Technique Used & Section & References \\ \hline
     AE  & Section~\ref{sec:ae} & ~\cite{wang2016research,cowton2018combined},~\cite{sato2018primitive}\\\hline
     DBN & Section~\ref{sec:dnn} & ~\cite{turner2014deep},~\cite{sharma2016abnormality},~\cite{wulsin2010semi},~\cite{ma2018unsupervised},~\cite{zhang2016automatic},~\cite{wulsin2011modeling} ,~\cite{wu2015adaptive}\\\hline
     RBM & Section~\ref{sec:dnn}  & ~\cite{liao2016enhanced}\\\hline
     VAE & Section~\ref{sec:gan_adversarial} & ~\cite{xu2018unsupervised},~\cite{lu2018anomaly} \\\hline
     GAN & Section~\ref{sec:gan_adversarial}&~\cite{ghasedi2018semi},~\cite{chen2018unsupervised} \\\hline
     LSTM ,RNN,GRU & Section~\ref{sec:rnn_lstm_gru} & ~\cite{yang2018toward},~\cite{jagannatha2016bidirectional},~\cite{cowton2018combined},~\cite{o2016recurrent},~\cite{latif2018phonocardiographic},~\cite{zhang2018time},~\cite{chauhan2015anomaly},~\cite{gugulothusparse,amarasinghe2018toward}\\\hline
     CNN  & Section~\ref{sec:cnn} & ~\cite{schmidt2018artificial},~\cite{esteva2017dermatologist},~\cite{wang2016research},~\cite{iakovidis2018detecting}\\\hline
     Hybrid( AE+ KNN) & Section~\ref{sec:cnn} & ~\cite{song2017hybrid} \\\hline
    \end{tabular}
\end{center}
\end{table*}
%%%%%%%%% End of table Medical anomaly detection







