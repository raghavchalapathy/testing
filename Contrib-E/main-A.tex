%%%%%%%%%%%%%%%%%%%%%%% file typeinst.tex %%%%%%%%%%%%%%%%%%%%%%%%%
%
% This is the LaTeX source for the instructions to authors using
% the LaTeX document class 'llncs.cls' for contributions to
% the Lecture Notes in Computer Sciences series.
% http://www.springer.com/lncs       Springer Heidelberg 2006/05/04
%
% It may be used as a template for your own input - copy it
% to a new file with a new name and use it as the basis
% for your article.
%
% NB: the document class 'llncs' has its own and detailed documentation, see
% ftp://ftp.springer.de/data/pubftp/pub/tex/latex/llncs/latex2e/llncsdoc.pdf
%
%%%%%%%%%%%%%%%%%%%%%%%%%%%%%%%%%%%%%%%%%%%%%%%%%%%%%%%%%%%%%%%%%%%
\documentclass[runningheads,a4paper]{llncs}
\usepackage{amsmath,amssymb}
\setcounter{tocdepth}{3}
\usepackage{graphicx}
\graphicspath{ {images/} }
\usepackage{multirow}
\usepackage{url}
\usepackage{verbatim}
%\usepackage[colorlinks=true,citecolor=blue]{hyperref}

\urldef{\mailsa}\path|rcha9612@uni.sydney.edu.au, etot5316@uni.sydney.edu.au, schawla@qf.org.qa|

\newcommand{\keywords}[1]{\par\addvspace\baselineskip
\noindent\keywordname\enspace\ignorespaces#1}
\usepackage{float}
\usepackage{subfigure}
\usepackage{multirow}
\usepackage{soul,color,colortbl}
\usepackage{xcolor}
\usepackage{booktabs}
\usepackage{upgreek}
% \usepackage{algorithm}
\usepackage[noend]{algpseudocode}
\usepackage[ruled,vlined,linesnumbered]{algorithm2e}

\usepackage{booktabs} % For formal tables
\usepackage[english]{babel}
\usepackage[utf8]{inputenc}
\usepackage{amssymb}

\newcommand{\A}{\mathbf{A}}
\newcommand{\B}{\mathbf{B}}
\newcommand{\E}{\mathbf{E}}
\newcommand{\I}{\mathbf{I}}
\newcommand{\J}{\mathbf{J}}
\newcommand{\N}{\mathbf{N}}
\newcommand{\R}{\mathbf{R}}
\renewcommand{\S}{\mathbf{S}}
\newcommand{\T}{\mathbf{T}}
\newcommand{\U}{\mathbf{U}}
\newcommand{\V}{\mathbf{V}}
\newcommand{\W}{\mathbf{W}}
\newcommand{\X}{\mathbf{X}}
\newcommand{\Z}{\mathbf{Z}}


\newcommand{\Real}{\mathbb{R}}
\newcommand{\sanjay}[1]{\hl{\footnote{\hl{Sanjay: #1}}}}

\let\oldFootnote\footnote
\newcommand\nextToken\relax

\renewcommand\footnote[1]{%
    \oldFootnote{#1}\futurelet\nextToken\isFootnote}

\newcommand\isFootnote{%
    \ifx\footnote\nextToken\textsuperscript{,}\fi}


\mainmatter  % start of an individual contribution

% first the title is needed
\title{Group Anomaly Detection using Deep Generative Models}

% a short form should be given in case it is too long for the running head
\titlerunning{Group Anomaly Detection using Deep Generative Models}



% the name(s) of the author(s) follow(s) next
%
% NB: Chinese authors should write their first names(s) in front of
% their surnames. This ensures that the names appear correctly in
% the running heads and the author index.
%
\usepackage[symbol]{footmisc}
  \setcounter{footnote}{0}
 \author{Raghavendra Chalapathy\inst{1} \footnote {Equal Contribution \label{foot1}} \and Edward Toth\inst{2 }   \footref{foot1} \and Sanjay Chawla\inst{3}}

%
 \authorrunning{Chalapathy, Toth and Chawla}
% (feature abused for this document to repeat the title also on left hand pages)

% the affiliations are given next; don't give your e-mail address
% unless you accept that it will be published
\institute{The University of Sydney and Capital Markets CRC
\and
School of Information Technologies, The University of Sydney
\and
Qatar Computing Research Institute, HBKU
}
% \mailsa\\

% NB: a more complex sample for affiliations and the mapping to the
% corresponding authors can be found in the file "llncs.dem"
% (search for the string "\mainmatter" where a contribution starts).
% "llncs.dem" accompanies the document class "llncs.cls".
%

\toctitle{Lecture Notes in Computer Science}
\tocauthor{Authors' Instructions}
\begin{document}
\maketitle
\begin{abstract}
Given a portfolio of stocks or a series of frames in a video how do we detect significant changes in a group of values for real-time applications? 
In this section,% we formalise the problem of sequentially detecting temporal changes in a group of stochastic processes.  As a solution to this particular problem,
 we propose the group temporal change (GT$\Delta$) algorithm, a simple yet effective technique for the sequential detection of significant changes in a variety of statistical properties of a group over time.  
 Due to the flexible framework of the GT$\Delta$ algorithm, a domain expert is able to select one or more statistical properties that they are interested in monitoring over a period of time. The usefulness of our proposed algorithm is also demonstrated against state-of-the-art techniques on synthetically generated data as well as on two real-world applications; a portfolio of healthcare stocks over a twenty year period and a video monitoring the solar flare activity of our Sun.

%Given a portfolio of stocks or a series of frames in a surveillance video, how do we detect dynamic temporal changes in their collective behaviors over a period of time? We focus on the online discovery of temporal changes when group memberships are known. % in an online setting.
% As a solution to this novel problem, we propose GRACE, an algorithm that is scalable, robust and flexible in its characterization of groups. %To the best of our knowledge, our method is the first online algorithm for detecting anomalous changes in dynamic groups.
%In this section, we simulate the theoretical distribution of test statistics computed in GRACE as well as conduct a robust comparison study with   state-of-the-art models. We also apply our proposed method on two real world datasets, a portfolio of health care stocks and video footage recording the movement of pedestrians.
%

\end{abstract}

\section{Anomaly detection: motivation and challenges}
%%%%%%%%%%%%%%%%%%%%%%%%%%%%%%%%%%%%%%%%%%%%%%%%%%%%%%%%%%%%%%%%%%%%%%%%%
%
%	LaTeX File for Stanford University PhD Thesis
%
%%%%%%%%%%%%%%%%%%%%%%%%%%%%%%%%%%%%%%%%%%%%%%%%%%%%%%%%%%%%%%%%%%%%%%%%%
%	Copyright 2001  by  Jung-Suk Goo    (goojs@gloworm.stanford.edu)
%%%%%%%%%%%%%%%%%%%%%%%%%%%%%%%%%%%%%%%%%%%%%%%%%%%%%%%%%%%%%%%%%%%%%%%%%

\chapter{Introduction}

%%%%%%%%%%%%%%%%%%%%%%%%%%%%%%%%%%%%%%%%%%%%%%%%%%%%%%%%%%%%%%%%%%%%%%%%
%\section{Introduction} 
% allow  = 	permit, let, authorize, grant, empower, enable, entitle, qualify, agrees, offer, provide, express, show, assign, allocate, produce, construct, create, generate, induce, instigate, promote

% consistent = steady, stable, constant, regular, even, uniform, orderly, unchanging, unvarying, unswerving, undeviating, unwavering, unfluctuating, homogeneous, true to type; dependable, reliable, unfailing, predictable, reliable

% detect = 	identify, distinguish, establish, deduce, determine, differentiate, discriminate, discern, separate, characterise, discover, uncover, find, find out, turn up,  expose, reveal

% use = utilize, make use of, avail oneself of, employ, work, operate, wield, ply, apply, manoeuvre, manipulate,

%% In the introduction - do not explain any methods that are central to the comparison study
% Why Should we care?
%Background on group analysis - define group and explain, why are group interesting?
%
%1. Background.
%In this part you have to make clear what the context is. Ideally, you should give an idea of the state-of-the art of the field the report is about. But keep it short: in my opinion this part should be less than a page long. 
%Application motivation

%At the close of 2012, the monetary value of the world stock market was about US\$55 trillion.
Pointwise anomaly and change detection focus on the study of individual data instances that do not conform with the expected pattern in a dataset. With the increasing availability of multifaceted information, there is a growing trend in research involving groups or collections of observations. For example,  
 Muandet et al.	\cite{OCSMM} possibly detect Higgs bosons as a group of collision events  in high energy particle physics while a group of multiple sensor networks in Chen and Yu \cite{chen2016collaborative}, allow for a robust  detection of  distributed denial-of-service attacks. Group deviation detection  techniques achieve fewer false positives than pointwise approaches as a greater number of observations occur in group applications. %provide a better characterization of group behaviors. 
  Many pointwise anomaly detection methods are also not compatible  in detecting  group deviations so we turn to more specialized techniques.  
 



 

Group deviation detection involves the discovery of group behaviors which significantly deviate from the expected group patterns.   In particular, group anomaly detection (GAD) is the process of identifying groups that are not consistent with regular group patterns while group change detection (GCD) estimates significant deviations in the state of a group over  time. GAD and GCD methods achieve a higher performance  than pointwise methods for detecting group deviations. Even though GAD usually involves time-independent  applications and GCD relates to time-dependent groups, both problems share a common framework  and similar fundamental ideas.  
This survey  elaborates on group deviation detection techniques in static and dynamic situations. 

%Groups are also synonymous with collections, clusters or communities. 
%The terms group anomalies  and group outliers are interchangeable however we use group anomalies or anomalous groups in this survey.
 
% are synonymous with group outliers but  are also a specific type of collective contextual anomalies.    %Groups are defined according to different  contexts  where a group is a cluster of galexies in an astronomical dataset %\cite{OCSMM} ,MGM,FGM,GLAD}.
  %Group anomaly detection is the process of discovering patterns in groups that are not consistent with the expected behavior \cite{Chandola}. 

% emerging area of research


%Imagine a series of events leads to a financial butterfly effect and many portfolios of  stocks in the market require immediate asset reallocation. 


%groups exhibiting irregular behavior may represent a disease outbreaks  to malicious web spammers.  

  

%	  politics \cite{GLAD} 
  %Images in a photo album & Distortion \\

 
 
%Many papers mainly analyze the static natures of groups, however it is also interesting to monitor the evolution of groups over time. This is where multivariate time series and changepoint detection are applied.

% Group Distribution
%In our study, we focus on group behaviors based on numerical data. 




% \subsection{Definition of Groups} %
A group is a collection of two or more related data instances.    %F\~{a}rber et al. \cite{ClusterEval} even discusses how known group labels may not correspond to inherently clustered points. 
In GAD, a group anomaly has  significantly different  statistical properties  with respect to multiple groups whereas GCD involves detecting  a significant change   in a group  with respect to past group observations.   Group structures may be known a priori such as words in a documents otherwise when group   memberships between instances in a dataset are unknown, additional information or clustering algorithms are required. 
  Thus the initial definition of groups affects subsequent analysis and results. 
  %The validity of interpretations of known group labels and inherent clusters is discussed in  .   



 
    
   
More specifically,  Xiong et al.  \cite{MGM} categorize group deviations as either point-based or distribution-based for GAD applications.   Point-based anomalous groups are where all of the members are also pointwise anomalies. Similarly in GCD, a point-based group change signifies that time series in a group over time also experience significant changes.   On the other hand, a  distribution-based group anomaly  is where a collection of points differs from expected group patterns, however individual data instances may not seem anomalous. Likewise, a distribution-based change in a group over time occurs when individual time series exhibit regular behavior however their collective pattern is significantly different.   

 % between multiple variables. 
  
%  %  groups are represented by a set of features. %descriptive properties. 
%One way of specifying features of a group is through statistical properties of group distributions  such as location, scale, shape and dependence between multiple variables.   %Other  descriptive properties of a group % distributions 
%Other descriptions of group features such as rules and network connections are dependent on the availability of data. 
 
 

  %Their study investigates examples of Gaussian mixtures where a group anomaly is generated from a different proportion of distributions. 

%In these cases,  individual data instances exhibit regular behavior however the collective behaviors of roup 3 are anomalous in Figure \ref{Fig:Intro}. Since there are a variety of statistical properties for a group distribution, it is difficult to robustly capture anomalous behaviors using pointwise anomaly detection methods.





 %We focus on measures of distributions in terms of location, scale, shape and dependence between multiple variables. Different detection techniques have a better performance for particular anomalous patterns. 
  %  Thus our research analyzes a variety of  statistical properties as described in Table \ref{Tab:Des}. Most group anomaly detection methods are interested in distribution-based anomalies which are 




%\begin{figure}[H]
%\centering
% \begin{subfigure}[b]{0.6\textwidth}
%                \includegraphics[width=\linewidth,trim=3cm 12cm 3cm 11.5cm]{Ex1}
%                \caption{A significant deviation in scale or shape.}
%        \end{subfigure}%
%        \hfill
% \begin{subfigure}[b]{0.6\textwidth}
%                \includegraphics[width=\linewidth,trim=3cm 12cm 3cm 11.5cm]{Ex2}
%                \caption{Significant deviation in covariance or correlation. }
%  \end{subfigure}%
%%\includegraphics[width=7.5cm, height=5.2cm,trim=3cm 9cm 3cm 10cm]{ToyExample}
%\caption{Examples of group behaviors that clearly deviate in terms of different statistical properties; (a) scale or shape and (b) covariance or correlation.  
%}
%\label{Fig:Intro}
%\end{figure} 


 Figure \ref{Fig:Intro} highlights two examples where distribution-based group deviations are characterized by two different statistical properties. The first row in (a) illustrates Group 3 with a relatively greater  scale or shape whereas in the second row (b),  a Group 3 is characterized by a rotated covariance or correlation structure between variables. 
 % Depending on the context, Group 3 in each row of Figure \ref{Fig:Intro} may represent   a group anomaly for a GAD application whereas in a GCD context, Group 3 is a time-dependent group where a significant change in group occurs at $t=3$.  
  {  
  Depending on the specific domain, the group deviation (Group 3) in  Figure \ref{Fig:Intro} has different interpretations. In Xiong et al. \cite{MGM}, a group anomaly represents an anomalous galaxy with a significantly different scale or shape parameter whereas Chen et al. \cite{GLETS}  examine a GCD context where a significant deviation in correlation between stocks occurs over time.     
    }

%\subsection{Applications}
%By considering a group rather than an individual instance is  beneficial in a diverse range of applications.
 GAD and GCD  techniques provide meaningful insights that are not effectively detected by pointwise  methods  in a diverse range of applications.  
In intrusion detection,  Chen and Yu \cite{chen2016collaborative} explore  a collaborative detection system involving multiple sensor networks to detect distributed denial-of-service attacks. Using collective information from multiple intrusion detection system rather than a single system offers a more reliable detection of coordinated attacks. Another example is where Dai et al. \cite{ERACD} analyze an IMDb movie database where anomalous collection of entities contain highly ranked actors that are otherwise not discovered by pointwise detection methods.   Real-world GAD events have also been studied in group psychology such as  high  performance of employee work teams by   Kozlowski and Bell  \cite{kozlowski2003}.  
   %Zhou et al. \cite{Zhou2010}  A Survey of Coordinated Attacks and Collaborative Intrusion Detection}

Investigating group deviations has a variety of interesting domains, especially physical GAD applications.  %that motivate different avenues of research.
In particular,  Muandet et al.	\cite{OCSMM} investigate GAD for physical phenomena in high energy particle physics such as Higgs bosons that are observed as slight excesses in a collection of collision events rather than individual  events. In Guevara et al. \cite{SMDD}, an anomalous galaxy cluster is identified by an irregular proportion of color pixels. Xiong et al. \cite{FGM} also analyze a physical application with 3-dimensional velocity of a fluid from the  JHU turbulence database  
where a group anomaly represents unusual vorticity in  fluid dynamics.  
 

Textual data is also examined in GAD where a document is considered a group of words. 
Yu et al. \cite{GLAD} investigate scientific publications in order to understand the structure of certain research communities. Irregular communities of co-authors possibly reveal unusual research trends.  By analyzing documents from a training set of news articles, Soleimani and Miller \cite{ATD} infer regular topics   such as $`\mathtt{rec.sport.baseball}'$  and $`\mathtt{ talk.politics.misc}'$. An anomalous cluster in this case consists of novel topics that are unobserved in the training corpus such as $`\mathtt{rec.sport.hockey}'$ and $`\mathtt{talk.politics.mideast}'$. Using  textual information from product reviews on Amazon, Mukherjee et al.  \cite{GroupReviewSpam} identify  groups of spammers that collaborative in writing fake reviews. 

%There are many other applications where GAD and GCD techniques offer interesting results. 
  
  A group over time for GCD is also studied across a variety of domains.  Wong et al. \cite{wong-rule} investigate different demographic groups admitted to  emergency departments in hospitals in a major US city.   A significant change in a particular demographic group over time represents an early indication of a potential epidemic and disease outbreak. 
  Chen et al. \cite{GLETS} monitor a group of  time series  in the stock market where after a specific period, the group disbands with dissimilar individual behaviors.     In this case, a group of seemingly uncorrelated time series may also form a more cohesive collection with a higher correlation over time.  Using multiple sensor data, Xie and Siegmund  \cite{xie2013} explore a general problem of sequential change detection in a proportion of time series in a group over time.   In a political application, Yu et al. \cite{GLAD} discover  a large deviation in voting behaviors of a group of US senators around the time of a Democratic party  election.  
  A real-world GCD event has been discovered in five of the largest private health insurers in Chile where they colluded to unfairly reduced  the coverage of healthcare plans over a period of time \cite{Chile}.  

%Thus once a group anomaly is discovered,  an actionable intervention depending on the particular domain may lead to  mitigating health  risks to reduction of unfair monetary losses. 
 Thus there are interesting and meaningful insights that are gained from GAD and GCD applications such as: 
\begin{enumerate}
\item  New research discoveries; 
Higgs bosons in physics \cite{OCSMM}, 
 anomalous galaxy clusters in astronomy  \cite{SMDD},  unusual vorticity in fluid dynamics \cite{FGM}. 
\item  Mitigation of risks: reduce financial losses  \cite{GLETS}, prevent disease outbreaks  \cite{wong-rule}.
\item   Identification of  fraudulent collaborative  activities:  collusion detection \cite{Chile}, fake product reviews on Amazon  \cite{GroupReviewSpam}, % identifies collaborative group of spammers that write .
 intrusion detection for  distributed denial-of-service attacks  \cite{chen2016collaborative}.   
\item  Interesting  explanatory results;   research trends in academic communities   \cite{GLAD},  changes in political voting preferences \cite{GLAD},  
highly ranked actors in IMDb movies   \cite{ERACD}, performance of employee work teams  \cite{kozlowski2003}.  
\end{enumerate}  

   % Table \ref{Tab:Examples} summarizes group applications explored in the literature and provide a basic interpretations of their results.   
   
% 		\begin{table}[H]
%	\tabcolsep=0.2cm  	\renewcommand{\arraystretch}{1.8}
%	\begin{center}
%	\scalebox{0.8}{
%	\begin{tabular}{|p{3.5cm}|c|l|l|l|l }
%	\hline\\[-5mm]
%%Techniques &	
%Authors & \small Application & Group Dataset & Interpretation of  Results %Group Anomalies
%  \\ \hline \\[-5mm] 
%	% Molecular biology  & Irregular protein-protein interaction \cite{MMSB} \\
%%Discriminative Model  & 	
% \small Chen and Yu \cite{chen2016collaborative}&   & Collaborative  Intrusion Detection % System
%    &
%  Distributed denial-of-service   \\
% Muandet et al.	\cite{OCSMM} &   & High Energy Particle Physics  & Signals  containing  Higgs bosons  \\
% Guevara et al.	\cite{SMDD} &  & Sloan Digital Sky Survey  & Irregular cluster of galaxies  \\
%%\hline\\[-6mm] 
%%\small Xiong et al. \cite{MGM}  &	 Sloan Digital Sky Survey & Irregular cluster of galaxies  \\
%% 	\small Xiong et al. \cite{FGM} & Generative Model &
%% Generative Model  &
%\small Xiong et al. \cite{FGM} & GAD & JHU Turbulence Database Cluster  & Unusual vorticity \\%Image of Fluid Motion   & Unusual turbulence   \\ % Image Data  & Stitched images from different scenes  \\
%	\small Yu et al. \cite{GLAD} &  & Scientific Publications & Research trends in communities  \\
%Mukherjee et al.  \cite{GroupReviewSpam} &  &  Amazon reviews  & Groups of manipulative spammers \\ 
% \small Soleimani and Miller \cite{ATD}  &  & 20-Newsgroup Dataset  & Anomalous document collection   
% \\
% 	\small Dai et al. \cite{ERACD} &  &
%	IMDb Movie Database & Highly ranked actors  
%  \\   
%%Kozlowski et al. \cite{kozlowski2003}  
%\hline 
%	\small Yu et al. \cite{GLAD} &  & Political Voting & Changing voting preferences \\
%  \small Wong et al. \cite{wong-rule} 
%& \multirow{2}{*}{GCD} & Emergency Department %Database 
%	 &  Disease outbreaks \\
%     \small Chen et al. \cite{}  &   &  Stock Market Data     & Increased market variability   \\
%  Xie and Siegmund  \cite{xie2013} &   & Sequential Sensor Data & Change detection  in multiple sensors    \\ 
%     [2mm]
%   %Detecting Extreme Rank Anomalous Collections 
%% %Web Host Graph & Web Spammer entities \\
%% 
% \hline
%	 \end{tabular}
%	 }
%	 \smallskip
%	\end{center}
%	 with interpretations of group devi\caption{ Previous studies involving GAD and GCD applicationsations. }
%\label{Tab:Examples}
%\end{table}   


{ 
%\subsection{Our Contributions}
The objective of this survey paper is to provide a clearer understanding and detailed  overview of group deviation detection research.  %anomaly and change detection techniques involving group observations.  
 We first explain GAD techniques  in multiple static groups and then explore dynamic groups for GCD applications. %We also provide an evaluation of each procedure and suggest future research directions where techniques can be further improved. % Since some methods are  specifically design for a particular domain, we elaborate on the different applications for group anomaly detection.
Our  main contributions are  summarized as: %\vspace{-1mm}
\begin{enumerate}
\item {\bf Clearer Understanding:} 
This survey provides an underlying structure for   both group anomaly detection (GAD) and group change detection (GCD) problems.      % Figure \ref{Fig:Framework}  builds upon the anomaly detection from Chandola \cite{Chandola}
\item {\bf Detailed Overview:} 
 We further elucidate the details of state-of-the-art techniques in terms of four key components as described in Section \ref{Sec:Problem}. 
\item {\bf Discussion:} We also discuss the advantages and disadvantages of current GAD and GCD techniques in terms of discriminative methods, generative models  as well as hypothesis tests. 
\end{enumerate}
 
%\section{Organisation}
The rest of the paper is organized as follows. Section \ref{Sec:Framework} describes the underlying structure and ideas relating to group deviation detection.  Section \ref{Sec:Problem}  formalizes the group deviation detection problem where techniques are explained in terms of four key components. Techniques for detecting group anomalies are explained in Section  \ref{Sec:D} while Section \ref{Sec:GCD}  describes methods for detecting significant changes in a group over time. A discussion of our findings and future research for group deviation detection is provided in Section 
 \ref{Sec:Discussion} while Section \ref{Sec:Conclusion} summarizes our survey paper.
}
 

\section{Background and related work on group anomaly detection}
%\label{sec:background}
\label{sec:related}
% !TEX root=../main.tex
\section{Related Work} \label{DGM:RelatedWork}
GAD is an emerging area  of research where most state-of-the-art techniques have been more recently developed.  While group anomalies are briefly discussed in anomaly detection surveys such as Chandola et al. \cite{Chandola} and Austin \cite{Hodge},  
  Yu et al. \cite{SurveySocialMedia} elaborates on GAD techniques where group memberships are not previously known.  
   {  Recently Toth and Chawla \cite{MySurvey} provide a comprehensive overview of GAD methods as well as a detailed  description of detecting temporal changes in groups over time. 
}
We focus on group anomalies in image applications where group memberships are known a priori. 

Previous studies on image anomaly detection % involving image   applications 
can be understood in terms of group anomalies. 
%Anomaly detection has been  previously studied in a variety of image classification applications. 
Quellec et al.  \cite{mammo} examine mammographic images  where point-based group anomalies represent potentially cancerous regions. Perera and Patel \cite{chairs} learn features from a collection of images containing regular chair objects and detect point-based group anomalies where chairs have abnormal shapes, colors and other irregular characteristics. On the other hand, Xiong et al. \cite{FGM} detect distribution-based group anomalies that are stitched images from scene categories (inside city, mountain or coast). %At a pixel level,   Xiong et al. \cite{MGM} apply GAD methods to detect anomalous galaxy clusters with irregular proportions of RGB pixels. 
We emphasise detecting distribution-based group anomalies rather than point-based anomalies in our subsequent  image applications.

The discovery of group anomalies is of interest to a number of diverse domains.   
  Muandet et al.	\cite{OCSMM} investigate GAD for physical phenomena in high energy particle physics where Higgs bosons are observed as slight excesses in a collection of collision events rather than individual  events. Xiong et al. \cite{FGM} analyse fluid dynamics  where a group anomaly represents unusual vorticity and turbulence in  fluid motion.  % In topic modeling,  Soleimani and Miller \cite{ATD} characterise documents by topics and anomalous clusters of documents are discovered by their irregular  topic mixtures. 
   By incorporating additional information from pairwise connection data, Yu et al. \cite{GLAD} find potentially irregular communities of co-authors in various research communities.
Thus there are many GAD application other than image anomaly detection.
 

A  related discipline to image anomaly detection is video anomaly detection where many DGMs are applied.  
 Sultani  et al.  \cite{survideos1} detect real-world anomalies such as burglary, fighting, vandalism and so on from  CCTV footage using deep learning methods.  %Many techniques involving deep learning architectures have been applied to detect temporal changes in a video surveillance application.    
% In a review, Kiran et al. \cite{survideos2} compare DGMs with different  convolution architectures for  video anomaly detection applications. 
Recent work ~\cite{schlegl2017unsupervised,xu2018unsupervised,an2015variational} illustrate the effectiveness of generative models for high-dimensional anomaly detection. Although, there are existing works that apply DGMs in image-related applications, we leverage  autoencoders for DGMs  to detect group anomalies in a variety of image experiments. 
 



\section{Preliminaries}
\label{sec:preliminaries}
% !TEX root=../main.tex
\section{Preliminaries} \label{sec:preliminaries}
In this section, state-of-the-art GAD techniques  as well as DGMs are described.  

\subsection{Mixture of Gaussian Mixture (MGM) Models  }
\label{sec:mgmm} 

A hierarchical generative approach MGM is proposed by Xiong et al. \cite{MGM} for GAD. The data generating process in MGM  assumes that groups follow different types of Gaussian mixtures. % where different types of regular mixture proportion are  possible.
 Visual features are extracted from images then an anomalous group is characterised by an irregular mixture of visual features.  MGM is useful for distinguishing multiple types of group behaviours however poor results are obtained when group observations do  not appropriately follow the assumed generative process.

\subsection{One-Class Support Measure Machines (OCSMM)}
\label{sec:ocsmm}
 Muandet et al. \cite{OCSMM} propose the discriminative method OCSMM to maximise the margin that separates regular and anomalous group behaviours.    Each image is characterised by extracted visual features then OCSMM applies mean embedding functions and separates groups using a parameterised hyperplane.  OCSMM classifies groups as regular or anomalous however careful model  selection is required.   

\subsection{One-Class Support Vector Machines (OCSVM) }
\label{sec:ocsvm}
 If a group can be  reduced into a  single vector then pointwise anomaly detection methods such as OCSVM % from Sch{\"o}lkopf et al.
  \cite{OCSVM} are applicable.    We follow a bag of features approach in Azhar et al. \cite{SIFT-OCSVM}, where $k$-means is applied to extracted visual features and centroids are clustered into histogram intervals before implementing OCSVM.  OCSVM separates data using a parametrised hyperplane similar to OCSMM however it may not accurately detect group anomalies if  initial group characterisations are inadequate.  


\subsection{Deep Generative Models  (DGMs)}
\label{sec:adversarialAE}
This section  describes %the mathematical background of
 DGMs that are applied for the GAD problem. %used for outlier detection.
Consider  $M$ groups  where the $m$th group is denoted as input $G_m$ with a reconstructed output ${\hat G}_m$.  Firstly an autoencoder consists of  encoder $f_\phi$ to embed  inputs to latent variables %representation
 and  decoder $g_\psi$ which reconstructs inputs.  Autoencoders aim to reduce reconstruction error between inputs and ouputs with 
${ L_r(G_{m},\hat G_{m} )} = ||{ G_m - \hat G_m }||^2  \hspace{0.5cm}  %\mbox{where } G_m \in \mathbb{R}^{N \times V}
$.  
%\label{eqn:aeloss} \end{equation}
Reconstruction errors are treated as anomaly scores where groups with significantly high errors are considered anomalous. 

% Variational autoencoder
\subsubsection{Variational Autoencoders (VAE)}
\label{sec:Vautoencoders}
 Using variational inference (VI),   variational autoencoders (VAE)~\cite{Kingma2013}  
 infer latent variables $z$ that are produced by encoder $f_\phi$ with  assumed prior  $P(G_m)$. The core idea is to learn $P(z)$ from $P(z|G_m)$ %where reconstruction error
with loss function 
\begin{equation}
{ L(G_m,\hat G_m)} = { L_r(G_m,\hat G_m)} + KL(f_\phi(z|G_m)\, || \, g_\psi(z))  %\hspace{0.5cm}  G_m \in \mathbb{R}^{N \times V}
\label{eqn:vaeloss}
\end{equation}
In order to optimise Kullback-Leibler (KL) divergence, VAE parametrises groups by vectors of means and standard deviations ($\boldsymbol \mu$,$\boldsymbol \sigma$).  
A new sample  is generated from parameters ($\boldsymbol \mu$,$\boldsymbol \sigma$)  and  
  decoder $g_\psi$ reconstructs group inputs.   VAE utilises reconstruction probabilities~\cite{an2015variational} or reconstruction error to compute anomaly scores.

\subsubsection{Adversarial Autoencoders (AAE)}
\label{sec:aae}
One of the main limitations of VAE is lack of closed-form analytical solution for the KL divergence term except for few distributions. Adversarial autoencoders (AAE)~\cite{makhzani2015adversarial} avoid using KL divergence by adopting adversarial learning, to characterise broader set of distributions.   Firstly AAE infers latent representation $z$  according to generator network $f_\phi(z|G_m)$ and decoder $g_\psi$ reconstructs input %with $\hat G_m$.
  from $z$.  The weights of encoder $f_\phi$ and decoder $g_\psi$ are updated by backpropagating the reconstruction error between $ G_m$ and $\hat G_m$. 
Secondly discriminator $D$ receives $z$ and %$z \sim f_\phi(z|G_m)$ and  
$z' \sim P(z)$ then computes reconstruction scores  with $D(z)$ and $D(z')$ respectively. %The loss incurred is minimised by backpropagating through the discriminator to update its weights. 
The loss functions for autoencoder (or generator) $L_G$ and %is composed of reconstruction errors along with the loss for
 discriminator $L_D$ are given by  
\begin{equation}
\begin{aligned}
{L_G} = \frac{1}{M'} \sum_{m=1}^{M'} \log D(z_m) \mbox{ \;  and \; } L_D = -\frac{1}{M'} \sum_{m=1}^{M'} \big [\log D(z'_m)+ \log(1- D(z_m)) \big ]
\end{aligned}
\label{eqn:aaeloss}
\end{equation}
where $M'$ is the minibatch size %while $z$ represents the latent code generated by encoder 
and $z'$ is a sample generated from the true prior $P(z)$.


\section{Problem and Model Formulation }
\label{sec:method}
% !TEX root=../main.tex
\section{Problem and Model Formulation} \label{sec:method}
%\subsection
{\bf Problem Definition:} 
{  The following formulation follows the problem definition introduced in %Toth and Chawla \cite{MySurvey}
Section \ref{Sec:Problem}.  
Suppose groups %$\mathcal{G} = \big\{  {\bf G}_m \big\} _{ m=1 }^M  $
$ \big\{  {\bf G}_1, {\bf G}_2,\dots, {\bf G}_M \big\}   $ are observed where $M$ is the number of groups and the $m$th group has group size  
$N_m$ with $V$-dimensional observations, that is 
 ${\bf G}_m = \big( X_{mnv}\big)
 \in \mathbb{R}^{N_m \times V} $. 
} 
 In GAD, the statistical properties of the $m$th group is captured by a characterisation function denoted by $f:  \mathbb{R}^{N_m \times V} \to \mathbb{R}^{D}$ where $D$ is the dimensionality on a transformed feature space. After a characterisation function is applied to a training dataset,  group information is combined using an aggregation function $g: \mathbb{R}^{M \times D} \to \mathbb{R}^{D}$.  A group reference is composed of characterisation and aggregation functions applied with 
\begin{align}
\mathcal{G}^{(ref)} = g \Big[ \big\{ f({\bf G}_{m} ) \big\}_{m=1}^M \Big]
\label{eqn:Gref}
\end{align}
Then a distance metric $d(\cdot,\cdot) \ge 0  $ is applied to measure the deviation of a particular group from the group reference function. The distance score $  d\Big(\mathcal{G}^{(ref)}  , {\bf G}_{m} \Big )$  quantifies the deviance of the $m$th group from the expected group pattern where larger values are associated with more anomalous groups.  
Group anomalies are effectively detected when characterisation function $f$ and aggregation function $g$  respectively capture properties of group distributions and appropriately combine information into a group reference. For example in an variational autoencoder setting, an encoder function $f$ characterises mean and standard deviation  of group distributions whereas {  decoder function $g$ reconstructs the original sample.
 %with $f\big( {\bf G}_m \big) = ( {\mu}_m,{\sigma}_m)   $ for $ m = 1,2,\dots,M $.
 Further descriptions of functions $f$ and $g$ for VAE and AAE are provided in Algorithm \ref{algo:gadVae}.
 }
 
  

\vspace{4mm}
% Algorithm
\begin{algorithm}[t]
\DontPrintSemicolon
\SetAlgoLined
\SetKwInOut{Input}{Input}\SetKwInOut{Output}{Output}
\Input{ Groups $  \big\{  { \bf G}_1, { \bf  G}_2,\dots,  {\bf  G}_M \big \}  $  where  ${\bf G}_m  =\big( X_{mnv}\big) \in \mathbb{R}^{N_m \times V} $
%\big( X_{ij}\big) Set of points \bf{X} = ${\{x_1,...,x_N\}}$, known groups $\mathcal{G} = \big( {\bf G}(m)  \big)_{ m \in \{ 1,2,\dots,M \} }  $}
}
\BlankLine
\Output{Group anomaly scores \textbf{S}  }
\BlankLine
%$f_\phi,g_\psi \gets $
Train AAE and VAE to obtain encoder $f_\phi$ and decoder $g_\psi$    \;
\BlankLine
  \Begin{
        \Switch{C}{
            \Case{(VAE)}{
%              \For{(m = 1 to M)}{
 %   			\BlankLine
      				$(\mu_m,\sigma_m) \sim f_\phi(z|{\bf G}_m)$  for $m=1,2,\dots,M$ \;
  %              }
           $(\mu,\sigma) = \frac{1}{M}\sum_{m=1}^{M}      (\mu_m,\sigma_m$)\;
               \BlankLine
    draw a sample from $z \sim \mathcal{N}(\mu,\,\sigma)$\;
  %  reconstruct sample using decoder $\mathcal{G}^{(ref)}=    g_\psi(\mathcal{G}|z)$\;
            }
            \Case{(AAE)}{
%                 \For{(m = 1 to M)}{
%     			\BlankLine
%       				$z_m =  f_\phi(G_m)$\;
%                     }
          draw a latent representation $z \sim f_\phi(z|{\bf G}_m)$ \; for $m=1,2,\dots,M$
           }
          }
             \For{(m = 1 to M)}{
              reconstruct sample using decoder $\mathcal{G}^{(ref)}=    g_\psi( {\bf G}_m|z)$\;   
            compute the score
        ${ s}_m =d\Big(\mathcal{G}^{(ref)}, {\bf G}_{m}   \Big ) $

        \BlankLine
    }
     sort scores in descending order  
     \textbf{S}$= \{s_{(M)} >\dots>s_{(1)} \}$\;
     %$\{s_{(m)} \}_{m=1}^M  $
%     \; with  $s_{(M)} >s_{(M-1)}  > ...>s_{(1)} $\;
higher scores indicate more anomalous groups
%    groups with higher scores,  are more anomalous.\;  
     % that are further away from $\mathcal{G}^{(ref)}$ 
   
        \textbf{return S}
    }
\caption{Group anomaly detection using deep generative models}
\label{algo:gadVae}
\end{algorithm}


%
\subsection{Training the model}
\label{sec:training}
The variational and adversarial autoencoder are trained according to the objective function given in Equation ~(\ref{eqn:vaeloss}), (\ref{eqn:aaeloss}) respectively. The objective functions of DGMs are optimised using the standard backpropagation algorithm. Given known  group memberships, AAE is fully trained on input groups to obtain a  representative group reference $\mathcal{G}^{(ref)}$ described in Equation (\ref{eqn:Gref}). While in case of VAE, $\mathcal{G}^{(ref)}$ is obtained by drawing samples using mean and standard deviation parameters that are inferred from group distributions as illustrated in Algorithm~\ref{algo:gadVae}.

%
\subsection{Predicting with the model}
In order to identify  group anomalies, the distance of a group from  the group reference   $\mathcal{G}^{(ref)}$ is computed. The output scores are sorted according to descending order where groups that are furthest from $\mathcal{G}^{(ref)}$ are considered most anomalous. One convenient property of DGMs is that the anomaly detector will be inductive, i.e.  it can generalise to unseen observations. One can interpret the model as learning a robust representation of  group distributions. An appropriate characterisation of groups results in more accurate detection  where any unseen  observations  either  lie within the reference group manifold or deviate from the expected group pattern. % within that group.








\section{Experimental setup}
\label{sec:experiment-setup}
\section{Experimental Setup}  \label{sec:experiment-setup}
% !TEX root=../main.tex
\subsection{Datasets}
The DDIExtraction 2013 shared task challenge from SemEval-2013 Task 9.1~\cite{segura2013semeval} has provided a benchmark corpus for DNR and DDI extraction. The corpus contains manually-annotated pharmacological substances and drug-drug interactions (DDIs) for a total of $18,502$ pharmacological substances and $5,028$ DDIs. It collates two distinct datasets: DDI-DrugBank and DDI-MedLine~\cite{herrero2013ddi}. Table~\ref{table2} summarizes the basic statistics of the training and test datasets used in our experiments. For proper comparison, we follow the same settings as \cite{segura2015exploring}, using the training data of the DNR task along with the test data for the DDI task for training and validation of DNR. We split this joint dataset into a training and validation sets with approximately $70\%$ of sentences for training and the remaining for validation.

\subsection{Evaluation Methodology}
Our models have been blindly evaluated on unseen DNR test data using the \textit{strict} evaluation metrics. With this evaluation, the predicted entities have to match the ground-truth entities exactly, both in boundary and class. To facilitate the replication of our experimental results, we have used a publicly-available library for the implementation\footnote{\tt https://github.com/raghavchalapathy/dnr} (i.e., the Theano neural network toolkit \cite{bergstra2010theano}). The experiments have been run over a range of values for the hyper-parameters, using the validation set for selection~\cite{bergstra2012random}. The hyper-parameters include the number of hidden-layer nodes, $H \in \{25, 50, 100\}$, the context window size, $s \in \{1, 3, 5\}$, and the embedding dimension, $d \in \{50, 100, 300, 500, 1000\}$. Two additional parameters, the learning and drop-out rates, were sampled from a uniform distribution in the range $[0.05, 0.1]$. The embedding and initial weight matrices were all sampled from the uniform distribution within range $[-1, 1]$. Early training stopping was set to $100$ epochs to mollify over-fitting, and the model that gave the best performance on the validation set was retained. The accuracy is reported in terms of micro-average F$_1$ score computed using the CoNLL score function~\cite{Nadeau:07}.



\section{Experimental results}
\label{sec:experiment-results}

\section{Experimental Results}
\label{sec:unsup_experiment-results}

In this section, we present experiments for three scenarios:
(a) non-inductive anomaly detection,
(b) inductive anomaly detection, and
(c) image denoising.

%\vspace{-0.3 cm}
%%%
\subsection{Non-inductive anomaly detection results}

We present results on the three datasets described in Section \ref{sec:experiment-setup}.


%%%
\subsubsection{{(1) {\tt restaurant} dataset}}
We work with the {\tt restaurant} video activity detection dataset~\cite{xiong2011direct},
and consider the problem of inferring the background of videos via removal of (anomalous) foreground pixels.
Estimating the background in videos is important for tasks such as anomalous activity detection.
It is however difficult because of the variability of the background (e.g. due to lighting conditions) and the presence of foreground objects such as moving objects and people.

For this experiment, we only compare the RPCA and RCAE methods, owing to a lack of ground truth labels.

\textbf{Parameter settings}.
For RPCA, rank $K$ = 64 is used.

Per the success of the Batch Normalization architecture~\cite{ioffe2015batch} and Exponential Linear Units~\cite{clevert2015fast}, we have found that convolutional+batch-normalization+elu layers provide a better representation of convolutional filters.
Hence, in this experiment the RCAE adopts four layers of (conv-batch-normalization-elu) in the encoder part and four layers of  (conv-batch-normalization-elu) in the decoder portion of the network.
RCAE network parameters such as (number of filter, filter size, strides) are chosen to be (16,3,1) for first and second layers and (32,3,1) for third and fourth layers of both encoder and decoder layers.

\begin{figure}[!t]
	\centering
	\subfigure[RCAE.]{\includegraphics[scale=0.325]{images/Restaurant/CAE_worst.jpg}}
	\subfigure[RPCA.]{\includegraphics[scale=0.325]{images/Restaurant/RPCA_worst.jpg}}
	\caption{Top anomalous images containing original image (people walking in the lobby) decomposed into background (lobby) and foreground (people), {\tt restaurant} dataset.}
	\label{fig:results-restaurant}
\end{figure}

%\vspace{-0.3 cm}
\textbf{Results}.
While there are no ground truth anomalies in this dataset, a qualitative analysis reveals RCAE to outperforms its counterparts in capturing the foreground objects.
Figure~\ref{fig:results-restaurant} compares the top 6 most anomalous images for RCAE and RPCA.
We see that the most anomalous images contain high foregound activity (which are recognised as anomalous).
Visually, we see that the background reconstruction produced by RPCA contains a few blemishes in some cases, while for RCAE the backgrounds are smooth.


%%%
\subsubsection{{(2) {\tt usps} dataset}}
From the {\tt usps} handwritten digit dataset,
we create a dataset
with a mixture of 220 images of \lq1\rq s, and 11 images of \lq7\rq, as in~\cite{xu2010robust}.
Intuitively, the latter images are treated as being anomalous, as the corresponding images have different characteristics to the majority of the training data. Each image is flattened as a row vector, yielding a 231 $\times$ 256 training matrix.

\textbf{Parameter settings}.
For SVD and RPCA methods, rank $K = 64$ is used.
For AE, the inputs are flattened images as a column vector of size 256,
and the hidden layer is a column vector of size  64 (matching the rank $K$).

For DRMF, we follow the settings of~\cite{xu2010robust}.
For RKPCA, we used a Gaussian kernel with bandwidth $0.01$, a cost parameter $C = 1$, and requested $60\%$ of the KPCA spectrum (which roughly selects 64 principal components).

For RCAE, we set two layers of convolution layers with the filter number to be 32, filter size to be 3$\times$3, with number of strides as $1$ and  rectified linear unit (ReLU) as activation with max-pooling layer of dimension 2$\times$2.

\textbf{Results}.
From Table~\ref{tbl:anomaly-results-summary}, we see that it is a near certainty for all \lq7\rq\, are accurately identified as outliers.
Figure~\ref{fig:usps-anomalies} shows the top anomalous images for RCAE, where indeed the \lq7\rq's are correctly placed at the top of the list.
By contrast, for RPCA there are also some \lq1\rq's placed at the top.

\begin{figure}[!t]
	\centering
	\subfigure[RCAE.]{\includegraphics[scale=0.9]{usps-anomalies-crop}}
	\quad
	\subfigure[RPCA.]{\includegraphics[scale=0.29]{rpca_usps}}
	\caption{Top anomalous images, {\tt usps} dataset.}
	\label{fig:usps-anomalies}
	%\vspace{-\baselineskip}
\end{figure}

\begin{table*}[!t]
	\centering
	\renewcommand{\arraystretch}{1.25}
	\resizebox{0.99\linewidth}{!}{
	\subfigure[{\tt usps}]{
		\begin{tabular}{lccc}
			\toprule
			\toprule
			\textbf{Methods} & \textbf{AUPRC} & \textbf{AUROC} & \textbf{P@10} \\
			\midrule
			RCAE  & \cellcolor{gray!25}{0.9614 $\pm$ 0.0025}&\cellcolor{gray!25}{0.9988$\pm$ 0.0243}&\cellcolor{gray!25}{0.9108 $\pm$ 0.0113} \\
			\midrule
			CAE & 0.7003 $\pm$ 0.0105 & 0.9712 $\pm$ 0.0002 & 0.8730 $\pm$ 0.0023\\
			AE  & 0.8533 $\pm$ 0.0023 & 0.9927 $\pm$ 0.0022 & 0.8108 $\pm$ 0.0003 \\
			\midrule
			RKPCA & 0.5340 $\pm$ 0.0262 & 0.9717 $\pm$ 0.0024 & 0.5250 $\pm$ 0.0307 \\
			DRMF  & 0.7737 $\pm$ 0.0351 & 0.9928 $\pm$ 0.0027 & 0.7150 $\pm$ 0.0342 \\
			RPCA  & 0.7893 $\pm$ 0.0195 & 0.9942 $\pm$ 0.0012 & 0.7250 $\pm$ 0.0323\\
			SVD   & 0.6091 $\pm$ 0.1263 & 0.9800 $\pm$ 0.0105 & 0.5600 $\pm$ 0.0249 \\
			\bottomrule
	\end{tabular}}%
	\quad
	\subfigure[{\tt cifar-10}]{
		\begin{tabular}{ccc}
			\toprule
			\toprule
			\textbf{AUPRC} & \textbf{AUROC} & \textbf{P@10} \\
			\midrule
			\cellcolor{gray!25}{0.9934 $\pm$ 0.0003}&\cellcolor{gray!25}{0.6255 $\pm$ 0.0055} &\cellcolor{gray!25}{0.8716 $\pm$ 0.0005} \\
			\midrule
			0.9011 $\pm$ 0.0000 & 0.6191 $\pm$ 0.0000 & 0.0000 $\pm$ 0.0000 \\
			0.9341 $\pm$ 0.0029 & 0.5260 $\pm$ 0.0003 & 0.2000 $\pm$ 0.0003 \\
			\midrule
			0.0557 $\pm$ 0.0037 & 0.5026 $\pm$ 0.0123 & 0.0550 $\pm$ 0.0185 \\
			0.0034 $\pm$ 0.0000 & 0.4847 $\pm$ 0.0000 & 0.0000 $\pm$ 0.0000 \\
			0.0036 $\pm$ 0.0000 & 0.5211 $\pm$ 0.0000 & 0.0000 $\pm$ 0.0000 \\
			0.0024 $\pm$ 0.0000 & 0.5299 $\pm$ 0.0000 & 0.0000 $\pm$ 0.0000 \\
			\bottomrule
	\end{tabular}}%
	}

	\caption{Comparison between the baseline (bottom four rows) and state-of-the-art systems (top three rows). Results are the mean and standard error of performance metrics over 20 random training set draws. Highlighted cells indicate best performer.}
	\label{tbl:anomaly-results-summary}
	%\vspace{-0.2in}
\end{table*}

\subsubsection{{(3) {\tt cifar-10} dataset}}
We create a dataset with anomalies
by combining 5000 random images of dogs and 50 images of cats, as illustrated in Figure~\ref{fig:results-cifar}.
In this scenario the cats are anomalies, and the goal is to detect all the cats in an unsupervised manner.

\textbf{Parameter settings}.
For SVD and RPCA methods, rank $K = 64$ is used.
We trained a three-hidden-layer autoencoder (AE) (1024-256-64-256-1024 neurons).
The middle hidden layer size is set to be same as rank $K = 64$, %(same as number of components used in SVD and RPCA),
and the model is trained using Adam~\cite{kingma2014adam}.
The decoding layer uses sigmoid function in order to capture the nonlinearity characteristics from  latent representations produced by the hidden layer.
Finally, we obtain the feature vector for each image by obtaining the latent representation from the hidden layer.

For RKPCA, we used a Gaussian kernel with bandwidth $5 \cdot 10^{-8}$, a cost parameter $C = 0.1$, and requested $55\%$ of the KPCA spectrum (which roughly selects 64 principal components).
The RKPCA runtime was prohibitive on the full sample (see Sec \ref{sec:runtime}), so we resorted to a subsample of 1000 dogs and 50 cats.

The RCAE architecture in this experiment is same as for {\tt restaurant}, containing
four layers of  (conv-batch-normalization-elu) in the encoder part
and four layers of  (conv-batch-normalization-elu) in the decoder
portion of the network. RCAE network parameters such as (number of filter, filter size, strides) are chosen to be (16,3,1)  for first and second layers and (32,3,1) for third and fourth layers of both encoder and decoder.

\begin{figure}[!t]
	\centering
	\subfigure[RCAE.]{\includegraphics[scale=0.95]{images/CIFAR_10/CAE_worst_inductive_crop}}
	\subfigure[RPCA.]{\includegraphics[scale=0.95]{images/CIFAR_10/RPCA_worst_inductive_crop}}
	\caption{Top anomalous images, %along with background and foreground,
	{\tt cifar-10} dataset.}
	\label{fig:results-cifar}
\end{figure}

\textbf{Results}.
From Table \ref{tbl:anomaly-results-summary},
RCAE clearly outperforms all existing state-of-the art methods in anomaly detection.
Note that basic CAE, with no robustness (effectively $\lambda = \infty$), is also outperformed by our method, indicating that it is crucial to explicitly handle anomalies with the $\N$ term.

Figure~\ref{fig:results-cifar} illustrates the most anomalous images for our RCAE method, compared to RPCA.
Owing to the latter involving learning a linear subspace, the model is unable to effectively distinguish cats from dogs;
by contrast, RCAE can effectively determine the manifold characterising most dogs, and identifies cats to be anomalous with respect to this.


%%%%%%%%%
\subsection{Inductive anomaly detection results}

We conduct an experiment to assess the detection of \emph{inductive} anomalies.
Recall that this is a capability of our RCAE model, but not e.g. RPCA.
We consider the following setup:
we train our model on 5000 dog images, and then evaluate it on a test set comprising 500 dogs and 50 cat images.
As before, we wish all methods to accurately determine the cats to be anomalies.

\begin{figure}[!t]
	\centering
	\subfigure[RCAE.]{\includegraphics[scale=0.31]{images/CIFAR_10/caeInductiveAnomalies_crop.jpg}}
	\subfigure[CAE.]{\includegraphics[scale=0.95]{images/CIFAR_10/caeInductiveAnomalies_crop_new.jpg}}
	\caption{Top inductive anomalous images, {\tt cifar-10} dataset.}
	\label{fig:results-cifar-inductive}
\end{figure}


Table~\ref{tbl:inductive-anomaly-results} summarises the detection performance for all the methods on this inductive task.
The lower values compared to Table~\ref{tbl:anomaly-results-summary} are indicative that the problem here is more challenging than anomaly detection on a single dataset;
nonetheless, we see that our RCAE method manages to convincingly outperform both the SVD and AE baselines.
This is confirmed qualitatively in Figure~\ref{fig:results-cifar-inductive}, where we see that RCAE correctly identifies many cats in the test set as anomalous, while the basic
%AE
CAE
method suffers.

\begin{table}[!t]
	\centering
	\renewcommand{\arraystretch}{1.25}
	\caption{Inductive anomaly detection results on {\tt cifar-10}. Note that RPCA and DRMF are inapplicable here. Highlighted cells indicate best performer.}
	\resizebox{0.99\linewidth}{!}{
	\begin{tabular}{@{}lccccc@{}}
		\toprule
		\toprule
		& \textbf{SVD} & \textbf{RKPCA} & \textbf{AE} & \textbf{CAE} & \textbf{RCAE} \\
		\toprule
		\bf{AUPRC} & 0.1752 $\pm$ 0.0051 & 0.1006 $\pm$ 0.0045 & 0.6200 $\pm$ 0.0005 & 0.6423 $\pm$ 0.0005 & \cellcolor{gray!25}{0.6908 $\pm$ 0.0001} \\
		\bf{AUROC} & 0.4997 $\pm$ 0.0066 & 0.4988 $\pm$ 0.0125 & 0.5007 $\pm$ 0.0010 & 0.4708 $\pm$ 0.0003 & \cellcolor{gray!25}{0.5576 $\pm$ 0.0005} \\
		\bf{P@10}  & 0.2150 $\pm$ 0.0310 & 0.0900 $\pm$ 0.0228 & 0.1086 $\pm$ 0.0001 & 0.2908 $\pm$ 0.0001 & \cellcolor{gray!25}{0.5986 $\pm$ 0.0001} \\
		\bottomrule
	\end{tabular}
	}
	\label{tbl:inductive-anomaly-results}
\end{table}


%%%
\subsection{Image denoising results}

Finally, we test the ability of the model to de-noise images, which is a form of anomaly detection on individual pixels (or more generally, features).
In this experiment, we train all models on a set of 5000 images of dogs from {\tt cifar-10}.
For each image, we then add salt-and-pepper noise at a rate of 10\%.
Our goal is to recover the original image as accurately as possible.

Figure~\ref{fig:results-cifar-injection} illustrates that the most anomalous images in the presence of noise
contain images of the variations of dog class images (e.g. containing person's face).
Further, Figure \ref{fig:denoising-results} illustrates for various methods the mean square error between the reconstructed and original images.
RCAE effectively suppresses the noise as evident from the low error.
The improvement over raw CAE is modest, but suggests that there is benefit to explicitly accounting for noise.

%\vspace{-0.3 cm}
\begin{figure}[!t]
	\centering
	\subfigure[RCAE.]{\includegraphics[scale=0.95]{images/CIFAR_10/CAE_worst_crop.jpg}}
	\subfigure[RPCA.]{\includegraphics[scale=0.95]{images/CIFAR_10/RPCA_worst_crop.jpg}}
	\caption{Top anomalous images in original form (first row), noisy form (second row),
	image denoising task on {\tt cifar-10}.}
	\label{fig:results-cifar-injection}
\end{figure}

\begin{figure}[!t]
	\centering
	\includegraphics[scale=0.5]{images/cifar_10_mse.pdf}
	\caption{Illustration of the mean square error boxplots obtained for various models on image denoising task, {\tt cifar-10} dataset.
		In this setting, RCAE suppresses the noise and detects the background and foreground images effectively.}
	\label{fig:denoising-results}
	%\vspace{-\baselineskip}
\end{figure}


%%%
\subsection{Comparison of training times}
\label{sec:runtime}

We remark finally that our RCAE method is comparable in training efficiency to existing methods.
For example, on the small-scale {\tt restaurant} dataset, it takes 1 minute to train RPCA,
and 8.5 minutes to train RKPCA,
compared with 10 minutes for our RCAE method.
The ability to leverage recent advances in deep learning as part of our optimisation (e.g.\ training models on a GPU) is we believe a salient feature of our approach.

We note that while the RKPCA method is fast to train on smaller datasets, on larger datasets it suffers from the $O(n^2)$ complexity of kernel methods;
for example, it takes over an hour to train on the {\tt cifar-10} dataset.
It is plausible that one could leverage recent advances in fast approximations of kernel methods~\cite{Lopez-Paz:2014}, and studying these would be of interest in future work.
Note that the issue of using a fixed kernel function would remain, however.


\section{Conclusion}
\label{sec:conclusion}
% !TEX root=../main.tex
%By modeling images as group anomalies
\section{Conclusion}
\label{sec:conclusion}
This paper has investigated the effectiveness of recurrent neural architectures, namely the Elman and Jordan networks and the bidirectional LSTM-CRF, for drug name recognition. The most appealing feature of these architectures is their ability to provide end-to-end recognition straight from text, sparing effort from laborious feature construction. To the best of our knowledge, ours is the first paper to explore RNNs for entity recognition from pharmacological text. The experimental results over the SemEval-2013 Task 9.1 benchmarks look promising, with the bidirectional LSTM-CRF ranking closely to the state of the art. A potential way to  further improve its performance would be to initialize its training with unsupervised word embeddings such as Word2Vec~\cite{Mikolov:13} and GloVe~\cite{Pennington:14}. This approach has proved effective in many other domains and still dispenses with expert annotation effort; we plan this exploration for the near future.






\bibliographystyle{splncs03}
\bibliography{MyBibFile,outlier}

\end{document}


%%%%%%%%%%%%%%%%%%%%%%% file typeinst.tex %%%%%%%%%%%%%%%%%%%%%%%%%
%
% This is the LaTeX source for the instructions to authors using
% the LaTeX document class 'llncs.cls' for contributions to
% the Lecture Notes in Computer Sciences series.
% http://www.springer.com/lncs       Springer Heidelberg 2006/05/04
%
% It may be used as a template for your own input - copy it
% to a new file with a new name and use it as the basis
% for your article.
%
% NB: the document class 'llncs' has its own and detailed documentation, see
% ftp://ftp.springer.de/data/pubftp/pub/tex/latex/llncs/latex2e/llncsdoc.pdf
%
%%%%%%%%%%%%%%%%%%%%%%%%%%%%%%%%%%%%%%%%%%%%%%%%%%%%%%%%%%%%%%%%%%%
\documentclass[runningheads,a4paper]{llncs}

%\usepackage[colorlinks=true,citecolor=blue]{hyperref}
\usepackage{amsmath,amssymb}
\setcounter{tocdepth}{3}

\usepackage{graphicx}
\graphicspath{ {images/} }

\usepackage{multirow}
%\usepackage{url}
\usepackage{verbatim}

\urldef{\mailsa}\path|rcha9612@uni.sydney.edu.au, aditya.menon@data61.csiro.au, schawla@qf.org.qa|

\newcommand{\keywords}[1]{\par\addvspace\baselineskip
\noindent\keywordname\enspace\ignorespaces#1}

\usepackage{subfigure}
\usepackage{multirow}
\usepackage{soul,color,colortbl}
\usepackage{xcolor}
\usepackage{booktabs}
\usepackage{upgreek}
% \usepackage{algorithm}
\usepackage[noend]{algpseudocode}
\usepackage[ruled,vlined,linesnumbered]{algorithm2e}
\usepackage[tableposition=top]{caption}
\usepackage{booktabs} % For formal tables
\usepackage[english]{babel}
\usepackage[utf8]{inputenc}
\usepackage{amssymb}

\newcommand{\A}{\mathbf{A}}
\newcommand{\B}{\mathbf{B}}
\newcommand{\E}{\mathbf{E}}
\newcommand{\I}{\mathbf{I}}
\newcommand{\J}{\mathbf{J}}
\newcommand{\N}{\mathbf{N}}
\newcommand{\R}{\mathbf{R}}
\renewcommand{\S}{\mathbf{S}}
\newcommand{\T}{\mathbf{T}}
\newcommand{\U}{\mathbf{U}}
\newcommand{\V}{\mathbf{V}}
\newcommand{\W}{\mathbf{W}}
\newcommand{\X}{\mathbf{X}}
\newcommand{\Z}{\mathbf{Z}}

\captionsetup[subfigure]{font=ninept,labelfont=bf,textfont=smallfont,singlelinecheck=off,justification=raggedright}

\newcommand{\Real}{\mathbb{R}}
\newcommand{\sanjay}[1]{\hl{\footnote{\hl{Sanjay: #1}}}}

\let\oldFootnote\footnote
\newcommand\nextToken\relax

\renewcommand\footnote[1]{%
    \oldFootnote{#1}\futurelet\nextToken\isFootnote}

\newcommand\isFootnote{%
    \ifx\footnote\nextToken\textsuperscript{,}\fi}

\begin{document}

\mainmatter  % start of an individual contribution

% first the title is needed
\title{Group Anomaly Detection using Deep Generative Models}

% a short form should be given in case it is too long for the running head
\titlerunning{Group Anomaly Detection using Deep Generative Models}

% the name(s) of the author(s) follow(s) next
%
% NB: Chinese authors should write their first names(s) in front of
% their surnames. This ensures that the names appear correctly in
% the running heads and the author index.
%
% \author{Raghavendra Chalapathy\inst{1} \and Aditya Krishna Menon\inst{2} \and Sanjay Chawla\inst{3}}

%
% \authorrunning{Chalapathy, Menon and Chawla}
% (feature abused for this document to repeat the title also on left hand pages)

% the affiliations are given next; don't give your e-mail address
% unless you accept that it will be published
% \institute{University of Sydney and Capital Markets Cooperative Research Centre (CMCRC)
% \and
% Data61/CSIRO and the Australian National University
% \and
% Qatar Computing Research Institute \\
% \mailsa\\


%
% NB: a more complex sample for affiliations and the mapping to the
% corresponding authors can be found in the file "llncs.dem"
% (search for the string "\mainmatter" where a contribution starts).
% "llncs.dem" accompanies the document class "llncs.cls".
%

\toctitle{Lecture Notes in Computer Science}
\tocauthor{Authors' Instructions}
\maketitle


\begin{abstract}
Given a portfolio of stocks or a series of frames in a video how do we detect significant changes in a group of values for real-time applications? 
In this section,% we formalise the problem of sequentially detecting temporal changes in a group of stochastic processes.  As a solution to this particular problem,
 we propose the group temporal change (GT$\Delta$) algorithm, a simple yet effective technique for the sequential detection of significant changes in a variety of statistical properties of a group over time.  
 Due to the flexible framework of the GT$\Delta$ algorithm, a domain expert is able to select one or more statistical properties that they are interested in monitoring over a period of time. The usefulness of our proposed algorithm is also demonstrated against state-of-the-art techniques on synthetically generated data as well as on two real-world applications; a portfolio of healthcare stocks over a twenty year period and a video monitoring the solar flare activity of our Sun.

%Given a portfolio of stocks or a series of frames in a surveillance video, how do we detect dynamic temporal changes in their collective behaviors over a period of time? We focus on the online discovery of temporal changes when group memberships are known. % in an online setting.
% As a solution to this novel problem, we propose GRACE, an algorithm that is scalable, robust and flexible in its characterization of groups. %To the best of our knowledge, our method is the first online algorithm for detecting anomalous changes in dynamic groups.
%In this section, we simulate the theoretical distribution of test statistics computed in GRACE as well as conduct a robust comparison study with   state-of-the-art models. We also apply our proposed method on two real world datasets, a portfolio of health care stocks and video footage recording the movement of pedestrians.
%

\end{abstract}

\section{Anomaly detection: motivation and challenges}
%%%%%%%%%%%%%%%%%%%%%%%%%%%%%%%%%%%%%%%%%%%%%%%%%%%%%%%%%%%%%%%%%%%%%%%%%
%
%	LaTeX File for Stanford University PhD Thesis
%
%%%%%%%%%%%%%%%%%%%%%%%%%%%%%%%%%%%%%%%%%%%%%%%%%%%%%%%%%%%%%%%%%%%%%%%%%
%	Copyright 2001  by  Jung-Suk Goo    (goojs@gloworm.stanford.edu)
%%%%%%%%%%%%%%%%%%%%%%%%%%%%%%%%%%%%%%%%%%%%%%%%%%%%%%%%%%%%%%%%%%%%%%%%%

\chapter{Introduction}

%%%%%%%%%%%%%%%%%%%%%%%%%%%%%%%%%%%%%%%%%%%%%%%%%%%%%%%%%%%%%%%%%%%%%%%%
%\section{Introduction} 
% allow  = 	permit, let, authorize, grant, empower, enable, entitle, qualify, agrees, offer, provide, express, show, assign, allocate, produce, construct, create, generate, induce, instigate, promote

% consistent = steady, stable, constant, regular, even, uniform, orderly, unchanging, unvarying, unswerving, undeviating, unwavering, unfluctuating, homogeneous, true to type; dependable, reliable, unfailing, predictable, reliable

% detect = 	identify, distinguish, establish, deduce, determine, differentiate, discriminate, discern, separate, characterise, discover, uncover, find, find out, turn up,  expose, reveal

% use = utilize, make use of, avail oneself of, employ, work, operate, wield, ply, apply, manoeuvre, manipulate,

%% In the introduction - do not explain any methods that are central to the comparison study
% Why Should we care?
%Background on group analysis - define group and explain, why are group interesting?
%
%1. Background.
%In this part you have to make clear what the context is. Ideally, you should give an idea of the state-of-the art of the field the report is about. But keep it short: in my opinion this part should be less than a page long. 
%Application motivation

%At the close of 2012, the monetary value of the world stock market was about US\$55 trillion.
Pointwise anomaly and change detection focus on the study of individual data instances that do not conform with the expected pattern in a dataset. With the increasing availability of multifaceted information, there is a growing trend in research involving groups or collections of observations. For example,  
 Muandet et al.	\cite{OCSMM} possibly detect Higgs bosons as a group of collision events  in high energy particle physics while a group of multiple sensor networks in Chen and Yu \cite{chen2016collaborative}, allow for a robust  detection of  distributed denial-of-service attacks. Group deviation detection  techniques achieve fewer false positives than pointwise approaches as a greater number of observations occur in group applications. %provide a better characterization of group behaviors. 
  Many pointwise anomaly detection methods are also not compatible  in detecting  group deviations so we turn to more specialized techniques.  
 



 

Group deviation detection involves the discovery of group behaviors which significantly deviate from the expected group patterns.   In particular, group anomaly detection (GAD) is the process of identifying groups that are not consistent with regular group patterns while group change detection (GCD) estimates significant deviations in the state of a group over  time. GAD and GCD methods achieve a higher performance  than pointwise methods for detecting group deviations. Even though GAD usually involves time-independent  applications and GCD relates to time-dependent groups, both problems share a common framework  and similar fundamental ideas.  
This survey  elaborates on group deviation detection techniques in static and dynamic situations. 

%Groups are also synonymous with collections, clusters or communities. 
%The terms group anomalies  and group outliers are interchangeable however we use group anomalies or anomalous groups in this survey.
 
% are synonymous with group outliers but  are also a specific type of collective contextual anomalies.    %Groups are defined according to different  contexts  where a group is a cluster of galexies in an astronomical dataset %\cite{OCSMM} ,MGM,FGM,GLAD}.
  %Group anomaly detection is the process of discovering patterns in groups that are not consistent with the expected behavior \cite{Chandola}. 

% emerging area of research


%Imagine a series of events leads to a financial butterfly effect and many portfolios of  stocks in the market require immediate asset reallocation. 


%groups exhibiting irregular behavior may represent a disease outbreaks  to malicious web spammers.  

  

%	  politics \cite{GLAD} 
  %Images in a photo album & Distortion \\

 
 
%Many papers mainly analyze the static natures of groups, however it is also interesting to monitor the evolution of groups over time. This is where multivariate time series and changepoint detection are applied.

% Group Distribution
%In our study, we focus on group behaviors based on numerical data. 




% \subsection{Definition of Groups} %
A group is a collection of two or more related data instances.    %F\~{a}rber et al. \cite{ClusterEval} even discusses how known group labels may not correspond to inherently clustered points. 
In GAD, a group anomaly has  significantly different  statistical properties  with respect to multiple groups whereas GCD involves detecting  a significant change   in a group  with respect to past group observations.   Group structures may be known a priori such as words in a documents otherwise when group   memberships between instances in a dataset are unknown, additional information or clustering algorithms are required. 
  Thus the initial definition of groups affects subsequent analysis and results. 
  %The validity of interpretations of known group labels and inherent clusters is discussed in  .   



 
    
   
More specifically,  Xiong et al.  \cite{MGM} categorize group deviations as either point-based or distribution-based for GAD applications.   Point-based anomalous groups are where all of the members are also pointwise anomalies. Similarly in GCD, a point-based group change signifies that time series in a group over time also experience significant changes.   On the other hand, a  distribution-based group anomaly  is where a collection of points differs from expected group patterns, however individual data instances may not seem anomalous. Likewise, a distribution-based change in a group over time occurs when individual time series exhibit regular behavior however their collective pattern is significantly different.   

 % between multiple variables. 
  
%  %  groups are represented by a set of features. %descriptive properties. 
%One way of specifying features of a group is through statistical properties of group distributions  such as location, scale, shape and dependence between multiple variables.   %Other  descriptive properties of a group % distributions 
%Other descriptions of group features such as rules and network connections are dependent on the availability of data. 
 
 

  %Their study investigates examples of Gaussian mixtures where a group anomaly is generated from a different proportion of distributions. 

%In these cases,  individual data instances exhibit regular behavior however the collective behaviors of roup 3 are anomalous in Figure \ref{Fig:Intro}. Since there are a variety of statistical properties for a group distribution, it is difficult to robustly capture anomalous behaviors using pointwise anomaly detection methods.





 %We focus on measures of distributions in terms of location, scale, shape and dependence between multiple variables. Different detection techniques have a better performance for particular anomalous patterns. 
  %  Thus our research analyzes a variety of  statistical properties as described in Table \ref{Tab:Des}. Most group anomaly detection methods are interested in distribution-based anomalies which are 




%\begin{figure}[H]
%\centering
% \begin{subfigure}[b]{0.6\textwidth}
%                \includegraphics[width=\linewidth,trim=3cm 12cm 3cm 11.5cm]{Ex1}
%                \caption{A significant deviation in scale or shape.}
%        \end{subfigure}%
%        \hfill
% \begin{subfigure}[b]{0.6\textwidth}
%                \includegraphics[width=\linewidth,trim=3cm 12cm 3cm 11.5cm]{Ex2}
%                \caption{Significant deviation in covariance or correlation. }
%  \end{subfigure}%
%%\includegraphics[width=7.5cm, height=5.2cm,trim=3cm 9cm 3cm 10cm]{ToyExample}
%\caption{Examples of group behaviors that clearly deviate in terms of different statistical properties; (a) scale or shape and (b) covariance or correlation.  
%}
%\label{Fig:Intro}
%\end{figure} 


 Figure \ref{Fig:Intro} highlights two examples where distribution-based group deviations are characterized by two different statistical properties. The first row in (a) illustrates Group 3 with a relatively greater  scale or shape whereas in the second row (b),  a Group 3 is characterized by a rotated covariance or correlation structure between variables. 
 % Depending on the context, Group 3 in each row of Figure \ref{Fig:Intro} may represent   a group anomaly for a GAD application whereas in a GCD context, Group 3 is a time-dependent group where a significant change in group occurs at $t=3$.  
  {  
  Depending on the specific domain, the group deviation (Group 3) in  Figure \ref{Fig:Intro} has different interpretations. In Xiong et al. \cite{MGM}, a group anomaly represents an anomalous galaxy with a significantly different scale or shape parameter whereas Chen et al. \cite{GLETS}  examine a GCD context where a significant deviation in correlation between stocks occurs over time.     
    }

%\subsection{Applications}
%By considering a group rather than an individual instance is  beneficial in a diverse range of applications.
 GAD and GCD  techniques provide meaningful insights that are not effectively detected by pointwise  methods  in a diverse range of applications.  
In intrusion detection,  Chen and Yu \cite{chen2016collaborative} explore  a collaborative detection system involving multiple sensor networks to detect distributed denial-of-service attacks. Using collective information from multiple intrusion detection system rather than a single system offers a more reliable detection of coordinated attacks. Another example is where Dai et al. \cite{ERACD} analyze an IMDb movie database where anomalous collection of entities contain highly ranked actors that are otherwise not discovered by pointwise detection methods.   Real-world GAD events have also been studied in group psychology such as  high  performance of employee work teams by   Kozlowski and Bell  \cite{kozlowski2003}.  
   %Zhou et al. \cite{Zhou2010}  A Survey of Coordinated Attacks and Collaborative Intrusion Detection}

Investigating group deviations has a variety of interesting domains, especially physical GAD applications.  %that motivate different avenues of research.
In particular,  Muandet et al.	\cite{OCSMM} investigate GAD for physical phenomena in high energy particle physics such as Higgs bosons that are observed as slight excesses in a collection of collision events rather than individual  events. In Guevara et al. \cite{SMDD}, an anomalous galaxy cluster is identified by an irregular proportion of color pixels. Xiong et al. \cite{FGM} also analyze a physical application with 3-dimensional velocity of a fluid from the  JHU turbulence database  
where a group anomaly represents unusual vorticity in  fluid dynamics.  
 

Textual data is also examined in GAD where a document is considered a group of words. 
Yu et al. \cite{GLAD} investigate scientific publications in order to understand the structure of certain research communities. Irregular communities of co-authors possibly reveal unusual research trends.  By analyzing documents from a training set of news articles, Soleimani and Miller \cite{ATD} infer regular topics   such as $`\mathtt{rec.sport.baseball}'$  and $`\mathtt{ talk.politics.misc}'$. An anomalous cluster in this case consists of novel topics that are unobserved in the training corpus such as $`\mathtt{rec.sport.hockey}'$ and $`\mathtt{talk.politics.mideast}'$. Using  textual information from product reviews on Amazon, Mukherjee et al.  \cite{GroupReviewSpam} identify  groups of spammers that collaborative in writing fake reviews. 

%There are many other applications where GAD and GCD techniques offer interesting results. 
  
  A group over time for GCD is also studied across a variety of domains.  Wong et al. \cite{wong-rule} investigate different demographic groups admitted to  emergency departments in hospitals in a major US city.   A significant change in a particular demographic group over time represents an early indication of a potential epidemic and disease outbreak. 
  Chen et al. \cite{GLETS} monitor a group of  time series  in the stock market where after a specific period, the group disbands with dissimilar individual behaviors.     In this case, a group of seemingly uncorrelated time series may also form a more cohesive collection with a higher correlation over time.  Using multiple sensor data, Xie and Siegmund  \cite{xie2013} explore a general problem of sequential change detection in a proportion of time series in a group over time.   In a political application, Yu et al. \cite{GLAD} discover  a large deviation in voting behaviors of a group of US senators around the time of a Democratic party  election.  
  A real-world GCD event has been discovered in five of the largest private health insurers in Chile where they colluded to unfairly reduced  the coverage of healthcare plans over a period of time \cite{Chile}.  

%Thus once a group anomaly is discovered,  an actionable intervention depending on the particular domain may lead to  mitigating health  risks to reduction of unfair monetary losses. 
 Thus there are interesting and meaningful insights that are gained from GAD and GCD applications such as: 
\begin{enumerate}
\item  New research discoveries; 
Higgs bosons in physics \cite{OCSMM}, 
 anomalous galaxy clusters in astronomy  \cite{SMDD},  unusual vorticity in fluid dynamics \cite{FGM}. 
\item  Mitigation of risks: reduce financial losses  \cite{GLETS}, prevent disease outbreaks  \cite{wong-rule}.
\item   Identification of  fraudulent collaborative  activities:  collusion detection \cite{Chile}, fake product reviews on Amazon  \cite{GroupReviewSpam}, % identifies collaborative group of spammers that write .
 intrusion detection for  distributed denial-of-service attacks  \cite{chen2016collaborative}.   
\item  Interesting  explanatory results;   research trends in academic communities   \cite{GLAD},  changes in political voting preferences \cite{GLAD},  
highly ranked actors in IMDb movies   \cite{ERACD}, performance of employee work teams  \cite{kozlowski2003}.  
\end{enumerate}  

   % Table \ref{Tab:Examples} summarizes group applications explored in the literature and provide a basic interpretations of their results.   
   
% 		\begin{table}[H]
%	\tabcolsep=0.2cm  	\renewcommand{\arraystretch}{1.8}
%	\begin{center}
%	\scalebox{0.8}{
%	\begin{tabular}{|p{3.5cm}|c|l|l|l|l }
%	\hline\\[-5mm]
%%Techniques &	
%Authors & \small Application & Group Dataset & Interpretation of  Results %Group Anomalies
%  \\ \hline \\[-5mm] 
%	% Molecular biology  & Irregular protein-protein interaction \cite{MMSB} \\
%%Discriminative Model  & 	
% \small Chen and Yu \cite{chen2016collaborative}&   & Collaborative  Intrusion Detection % System
%    &
%  Distributed denial-of-service   \\
% Muandet et al.	\cite{OCSMM} &   & High Energy Particle Physics  & Signals  containing  Higgs bosons  \\
% Guevara et al.	\cite{SMDD} &  & Sloan Digital Sky Survey  & Irregular cluster of galaxies  \\
%%\hline\\[-6mm] 
%%\small Xiong et al. \cite{MGM}  &	 Sloan Digital Sky Survey & Irregular cluster of galaxies  \\
%% 	\small Xiong et al. \cite{FGM} & Generative Model &
%% Generative Model  &
%\small Xiong et al. \cite{FGM} & GAD & JHU Turbulence Database Cluster  & Unusual vorticity \\%Image of Fluid Motion   & Unusual turbulence   \\ % Image Data  & Stitched images from different scenes  \\
%	\small Yu et al. \cite{GLAD} &  & Scientific Publications & Research trends in communities  \\
%Mukherjee et al.  \cite{GroupReviewSpam} &  &  Amazon reviews  & Groups of manipulative spammers \\ 
% \small Soleimani and Miller \cite{ATD}  &  & 20-Newsgroup Dataset  & Anomalous document collection   
% \\
% 	\small Dai et al. \cite{ERACD} &  &
%	IMDb Movie Database & Highly ranked actors  
%  \\   
%%Kozlowski et al. \cite{kozlowski2003}  
%\hline 
%	\small Yu et al. \cite{GLAD} &  & Political Voting & Changing voting preferences \\
%  \small Wong et al. \cite{wong-rule} 
%& \multirow{2}{*}{GCD} & Emergency Department %Database 
%	 &  Disease outbreaks \\
%     \small Chen et al. \cite{}  &   &  Stock Market Data     & Increased market variability   \\
%  Xie and Siegmund  \cite{xie2013} &   & Sequential Sensor Data & Change detection  in multiple sensors    \\ 
%     [2mm]
%   %Detecting Extreme Rank Anomalous Collections 
%% %Web Host Graph & Web Spammer entities \\
%% 
% \hline
%	 \end{tabular}
%	 }
%	 \smallskip
%	\end{center}
%	 with interpretations of group devi\caption{ Previous studies involving GAD and GCD applicationsations. }
%\label{Tab:Examples}
%\end{table}   


{ 
%\subsection{Our Contributions}
The objective of this survey paper is to provide a clearer understanding and detailed  overview of group deviation detection research.  %anomaly and change detection techniques involving group observations.  
 We first explain GAD techniques  in multiple static groups and then explore dynamic groups for GCD applications. %We also provide an evaluation of each procedure and suggest future research directions where techniques can be further improved. % Since some methods are  specifically design for a particular domain, we elaborate on the different applications for group anomaly detection.
Our  main contributions are  summarized as: %\vspace{-1mm}
\begin{enumerate}
\item {\bf Clearer Understanding:} 
This survey provides an underlying structure for   both group anomaly detection (GAD) and group change detection (GCD) problems.      % Figure \ref{Fig:Framework}  builds upon the anomaly detection from Chandola \cite{Chandola}
\item {\bf Detailed Overview:} 
 We further elucidate the details of state-of-the-art techniques in terms of four key components as described in Section \ref{Sec:Problem}. 
\item {\bf Discussion:} We also discuss the advantages and disadvantages of current GAD and GCD techniques in terms of discriminative methods, generative models  as well as hypothesis tests. 
\end{enumerate}
 
%\section{Organisation}
The rest of the paper is organized as follows. Section \ref{Sec:Framework} describes the underlying structure and ideas relating to group deviation detection.  Section \ref{Sec:Problem}  formalizes the group deviation detection problem where techniques are explained in terms of four key components. Techniques for detecting group anomalies are explained in Section  \ref{Sec:D} while Section \ref{Sec:GCD}  describes methods for detecting significant changes in a group over time. A discussion of our findings and future research for group deviation detection is provided in Section 
 \ref{Sec:Discussion} while Section \ref{Sec:Conclusion} summarizes our survey paper.
}
 

\section{Background and related work on group anomaly detection}
\label{sec:background}
\label{sec:related}
% !TEX root=../main.tex
\section{Related Work} \label{DGM:RelatedWork}
GAD is an emerging area  of research where most state-of-the-art techniques have been more recently developed.  While group anomalies are briefly discussed in anomaly detection surveys such as Chandola et al. \cite{Chandola} and Austin \cite{Hodge},  
  Yu et al. \cite{SurveySocialMedia} elaborates on GAD techniques where group memberships are not previously known.  
   {  Recently Toth and Chawla \cite{MySurvey} provide a comprehensive overview of GAD methods as well as a detailed  description of detecting temporal changes in groups over time. 
}
We focus on group anomalies in image applications where group memberships are known a priori. 

Previous studies on image anomaly detection % involving image   applications 
can be understood in terms of group anomalies. 
%Anomaly detection has been  previously studied in a variety of image classification applications. 
Quellec et al.  \cite{mammo} examine mammographic images  where point-based group anomalies represent potentially cancerous regions. Perera and Patel \cite{chairs} learn features from a collection of images containing regular chair objects and detect point-based group anomalies where chairs have abnormal shapes, colors and other irregular characteristics. On the other hand, Xiong et al. \cite{FGM} detect distribution-based group anomalies that are stitched images from scene categories (inside city, mountain or coast). %At a pixel level,   Xiong et al. \cite{MGM} apply GAD methods to detect anomalous galaxy clusters with irregular proportions of RGB pixels. 
We emphasise detecting distribution-based group anomalies rather than point-based anomalies in our subsequent  image applications.

The discovery of group anomalies is of interest to a number of diverse domains.   
  Muandet et al.	\cite{OCSMM} investigate GAD for physical phenomena in high energy particle physics where Higgs bosons are observed as slight excesses in a collection of collision events rather than individual  events. Xiong et al. \cite{FGM} analyse fluid dynamics  where a group anomaly represents unusual vorticity and turbulence in  fluid motion.  % In topic modeling,  Soleimani and Miller \cite{ATD} characterise documents by topics and anomalous clusters of documents are discovered by their irregular  topic mixtures. 
   By incorporating additional information from pairwise connection data, Yu et al. \cite{GLAD} find potentially irregular communities of co-authors in various research communities.
Thus there are many GAD application other than image anomaly detection.
 

A  related discipline to image anomaly detection is video anomaly detection where many DGMs are applied.  
 Sultani  et al.  \cite{survideos1} detect real-world anomalies such as burglary, fighting, vandalism and so on from  CCTV footage using deep learning methods.  %Many techniques involving deep learning architectures have been applied to detect temporal changes in a video surveillance application.    
% In a review, Kiran et al. \cite{survideos2} compare DGMs with different  convolution architectures for  video anomaly detection applications. 
Recent work ~\cite{schlegl2017unsupervised,xu2018unsupervised,an2015variational} illustrate the effectiveness of generative models for high-dimensional anomaly detection. Although, there are existing works that apply DGMs in image-related applications, we leverage  autoencoders for DGMs  to detect group anomalies in a variety of image experiments. 
 



\section{Preliminaries}
\label{sec:preliminaries}
% !TEX root=../main.tex
\section{Preliminaries} \label{sec:preliminaries}
In this section, state-of-the-art GAD techniques  as well as DGMs are described.  

\subsection{Mixture of Gaussian Mixture (MGM) Models  }
\label{sec:mgmm} 

A hierarchical generative approach MGM is proposed by Xiong et al. \cite{MGM} for GAD. The data generating process in MGM  assumes that groups follow different types of Gaussian mixtures. % where different types of regular mixture proportion are  possible.
 Visual features are extracted from images then an anomalous group is characterised by an irregular mixture of visual features.  MGM is useful for distinguishing multiple types of group behaviours however poor results are obtained when group observations do  not appropriately follow the assumed generative process.

\subsection{One-Class Support Measure Machines (OCSMM)}
\label{sec:ocsmm}
 Muandet et al. \cite{OCSMM} propose the discriminative method OCSMM to maximise the margin that separates regular and anomalous group behaviours.    Each image is characterised by extracted visual features then OCSMM applies mean embedding functions and separates groups using a parameterised hyperplane.  OCSMM classifies groups as regular or anomalous however careful model  selection is required.   

\subsection{One-Class Support Vector Machines (OCSVM) }
\label{sec:ocsvm}
 If a group can be  reduced into a  single vector then pointwise anomaly detection methods such as OCSVM % from Sch{\"o}lkopf et al.
  \cite{OCSVM} are applicable.    We follow a bag of features approach in Azhar et al. \cite{SIFT-OCSVM}, where $k$-means is applied to extracted visual features and centroids are clustered into histogram intervals before implementing OCSVM.  OCSVM separates data using a parametrised hyperplane similar to OCSMM however it may not accurately detect group anomalies if  initial group characterisations are inadequate.  


\subsection{Deep Generative Models  (DGMs)}
\label{sec:adversarialAE}
This section  describes %the mathematical background of
 DGMs that are applied for the GAD problem. %used for outlier detection.
Consider  $M$ groups  where the $m$th group is denoted as input $G_m$ with a reconstructed output ${\hat G}_m$.  Firstly an autoencoder consists of  encoder $f_\phi$ to embed  inputs to latent variables %representation
 and  decoder $g_\psi$ which reconstructs inputs.  Autoencoders aim to reduce reconstruction error between inputs and ouputs with 
${ L_r(G_{m},\hat G_{m} )} = ||{ G_m - \hat G_m }||^2  \hspace{0.5cm}  %\mbox{where } G_m \in \mathbb{R}^{N \times V}
$.  
%\label{eqn:aeloss} \end{equation}
Reconstruction errors are treated as anomaly scores where groups with significantly high errors are considered anomalous. 

% Variational autoencoder
\subsubsection{Variational Autoencoders (VAE)}
\label{sec:Vautoencoders}
 Using variational inference (VI),   variational autoencoders (VAE)~\cite{Kingma2013}  
 infer latent variables $z$ that are produced by encoder $f_\phi$ with  assumed prior  $P(G_m)$. The core idea is to learn $P(z)$ from $P(z|G_m)$ %where reconstruction error
with loss function 
\begin{equation}
{ L(G_m,\hat G_m)} = { L_r(G_m,\hat G_m)} + KL(f_\phi(z|G_m)\, || \, g_\psi(z))  %\hspace{0.5cm}  G_m \in \mathbb{R}^{N \times V}
\label{eqn:vaeloss}
\end{equation}
In order to optimise Kullback-Leibler (KL) divergence, VAE parametrises groups by vectors of means and standard deviations ($\boldsymbol \mu$,$\boldsymbol \sigma$).  
A new sample  is generated from parameters ($\boldsymbol \mu$,$\boldsymbol \sigma$)  and  
  decoder $g_\psi$ reconstructs group inputs.   VAE utilises reconstruction probabilities~\cite{an2015variational} or reconstruction error to compute anomaly scores.

\subsubsection{Adversarial Autoencoders (AAE)}
\label{sec:aae}
One of the main limitations of VAE is lack of closed-form analytical solution for the KL divergence term except for few distributions. Adversarial autoencoders (AAE)~\cite{makhzani2015adversarial} avoid using KL divergence by adopting adversarial learning, to characterise broader set of distributions.   Firstly AAE infers latent representation $z$  according to generator network $f_\phi(z|G_m)$ and decoder $g_\psi$ reconstructs input %with $\hat G_m$.
  from $z$.  The weights of encoder $f_\phi$ and decoder $g_\psi$ are updated by backpropagating the reconstruction error between $ G_m$ and $\hat G_m$. 
Secondly discriminator $D$ receives $z$ and %$z \sim f_\phi(z|G_m)$ and  
$z' \sim P(z)$ then computes reconstruction scores  with $D(z)$ and $D(z')$ respectively. %The loss incurred is minimised by backpropagating through the discriminator to update its weights. 
The loss functions for autoencoder (or generator) $L_G$ and %is composed of reconstruction errors along with the loss for
 discriminator $L_D$ are given by  
\begin{equation}
\begin{aligned}
{L_G} = \frac{1}{M'} \sum_{m=1}^{M'} \log D(z_m) \mbox{ \;  and \; } L_D = -\frac{1}{M'} \sum_{m=1}^{M'} \big [\log D(z'_m)+ \log(1- D(z_m)) \big ]
\end{aligned}
\label{eqn:aaeloss}
\end{equation}
where $M'$ is the minibatch size %while $z$ represents the latent code generated by encoder 
and $z'$ is a sample generated from the true prior $P(z)$.


\section{Model Formulation}
\label{sec:method}
% !TEX root=../main.tex
\section{Problem and Model Formulation} \label{sec:method}
%\subsection
{\bf Problem Definition:} 
{  The following formulation follows the problem definition introduced in %Toth and Chawla \cite{MySurvey}
Section \ref{Sec:Problem}.  
Suppose groups %$\mathcal{G} = \big\{  {\bf G}_m \big\} _{ m=1 }^M  $
$ \big\{  {\bf G}_1, {\bf G}_2,\dots, {\bf G}_M \big\}   $ are observed where $M$ is the number of groups and the $m$th group has group size  
$N_m$ with $V$-dimensional observations, that is 
 ${\bf G}_m = \big( X_{mnv}\big)
 \in \mathbb{R}^{N_m \times V} $. 
} 
 In GAD, the statistical properties of the $m$th group is captured by a characterisation function denoted by $f:  \mathbb{R}^{N_m \times V} \to \mathbb{R}^{D}$ where $D$ is the dimensionality on a transformed feature space. After a characterisation function is applied to a training dataset,  group information is combined using an aggregation function $g: \mathbb{R}^{M \times D} \to \mathbb{R}^{D}$.  A group reference is composed of characterisation and aggregation functions applied with 
\begin{align}
\mathcal{G}^{(ref)} = g \Big[ \big\{ f({\bf G}_{m} ) \big\}_{m=1}^M \Big]
\label{eqn:Gref}
\end{align}
Then a distance metric $d(\cdot,\cdot) \ge 0  $ is applied to measure the deviation of a particular group from the group reference function. The distance score $  d\Big(\mathcal{G}^{(ref)}  , {\bf G}_{m} \Big )$  quantifies the deviance of the $m$th group from the expected group pattern where larger values are associated with more anomalous groups.  
Group anomalies are effectively detected when characterisation function $f$ and aggregation function $g$  respectively capture properties of group distributions and appropriately combine information into a group reference. For example in an variational autoencoder setting, an encoder function $f$ characterises mean and standard deviation  of group distributions whereas {  decoder function $g$ reconstructs the original sample.
 %with $f\big( {\bf G}_m \big) = ( {\mu}_m,{\sigma}_m)   $ for $ m = 1,2,\dots,M $.
 Further descriptions of functions $f$ and $g$ for VAE and AAE are provided in Algorithm \ref{algo:gadVae}.
 }
 
  

\vspace{4mm}
% Algorithm
\begin{algorithm}[t]
\DontPrintSemicolon
\SetAlgoLined
\SetKwInOut{Input}{Input}\SetKwInOut{Output}{Output}
\Input{ Groups $  \big\{  { \bf G}_1, { \bf  G}_2,\dots,  {\bf  G}_M \big \}  $  where  ${\bf G}_m  =\big( X_{mnv}\big) \in \mathbb{R}^{N_m \times V} $
%\big( X_{ij}\big) Set of points \bf{X} = ${\{x_1,...,x_N\}}$, known groups $\mathcal{G} = \big( {\bf G}(m)  \big)_{ m \in \{ 1,2,\dots,M \} }  $}
}
\BlankLine
\Output{Group anomaly scores \textbf{S}  }
\BlankLine
%$f_\phi,g_\psi \gets $
Train AAE and VAE to obtain encoder $f_\phi$ and decoder $g_\psi$    \;
\BlankLine
  \Begin{
        \Switch{C}{
            \Case{(VAE)}{
%              \For{(m = 1 to M)}{
 %   			\BlankLine
      				$(\mu_m,\sigma_m) \sim f_\phi(z|{\bf G}_m)$  for $m=1,2,\dots,M$ \;
  %              }
           $(\mu,\sigma) = \frac{1}{M}\sum_{m=1}^{M}      (\mu_m,\sigma_m$)\;
               \BlankLine
    draw a sample from $z \sim \mathcal{N}(\mu,\,\sigma)$\;
  %  reconstruct sample using decoder $\mathcal{G}^{(ref)}=    g_\psi(\mathcal{G}|z)$\;
            }
            \Case{(AAE)}{
%                 \For{(m = 1 to M)}{
%     			\BlankLine
%       				$z_m =  f_\phi(G_m)$\;
%                     }
          draw a latent representation $z \sim f_\phi(z|{\bf G}_m)$ \; for $m=1,2,\dots,M$
           }
          }
             \For{(m = 1 to M)}{
              reconstruct sample using decoder $\mathcal{G}^{(ref)}=    g_\psi( {\bf G}_m|z)$\;   
            compute the score
        ${ s}_m =d\Big(\mathcal{G}^{(ref)}, {\bf G}_{m}   \Big ) $

        \BlankLine
    }
     sort scores in descending order  
     \textbf{S}$= \{s_{(M)} >\dots>s_{(1)} \}$\;
     %$\{s_{(m)} \}_{m=1}^M  $
%     \; with  $s_{(M)} >s_{(M-1)}  > ...>s_{(1)} $\;
higher scores indicate more anomalous groups
%    groups with higher scores,  are more anomalous.\;  
     % that are further away from $\mathcal{G}^{(ref)}$ 
   
        \textbf{return S}
    }
\caption{Group anomaly detection using deep generative models}
\label{algo:gadVae}
\end{algorithm}


%
\subsection{Training the model}
\label{sec:training}
The variational and adversarial autoencoder are trained according to the objective function given in Equation ~(\ref{eqn:vaeloss}), (\ref{eqn:aaeloss}) respectively. The objective functions of DGMs are optimised using the standard backpropagation algorithm. Given known  group memberships, AAE is fully trained on input groups to obtain a  representative group reference $\mathcal{G}^{(ref)}$ described in Equation (\ref{eqn:Gref}). While in case of VAE, $\mathcal{G}^{(ref)}$ is obtained by drawing samples using mean and standard deviation parameters that are inferred from group distributions as illustrated in Algorithm~\ref{algo:gadVae}.

%
\subsection{Predicting with the model}
In order to identify  group anomalies, the distance of a group from  the group reference   $\mathcal{G}^{(ref)}$ is computed. The output scores are sorted according to descending order where groups that are furthest from $\mathcal{G}^{(ref)}$ are considered most anomalous. One convenient property of DGMs is that the anomaly detector will be inductive, i.e.  it can generalise to unseen observations. One can interpret the model as learning a robust representation of  group distributions. An appropriate characterisation of groups results in more accurate detection  where any unseen  observations  either  lie within the reference group manifold or deviate from the expected group pattern. % within that group.








\section{Experimental setup}
\label{sec:experiment-setup}
\section{Experimental Setup}  \label{sec:experiment-setup}
% !TEX root=../main.tex
\subsection{Datasets}
The DDIExtraction 2013 shared task challenge from SemEval-2013 Task 9.1~\cite{segura2013semeval} has provided a benchmark corpus for DNR and DDI extraction. The corpus contains manually-annotated pharmacological substances and drug-drug interactions (DDIs) for a total of $18,502$ pharmacological substances and $5,028$ DDIs. It collates two distinct datasets: DDI-DrugBank and DDI-MedLine~\cite{herrero2013ddi}. Table~\ref{table2} summarizes the basic statistics of the training and test datasets used in our experiments. For proper comparison, we follow the same settings as \cite{segura2015exploring}, using the training data of the DNR task along with the test data for the DDI task for training and validation of DNR. We split this joint dataset into a training and validation sets with approximately $70\%$ of sentences for training and the remaining for validation.

\subsection{Evaluation Methodology}
Our models have been blindly evaluated on unseen DNR test data using the \textit{strict} evaluation metrics. With this evaluation, the predicted entities have to match the ground-truth entities exactly, both in boundary and class. To facilitate the replication of our experimental results, we have used a publicly-available library for the implementation\footnote{\tt https://github.com/raghavchalapathy/dnr} (i.e., the Theano neural network toolkit \cite{bergstra2010theano}). The experiments have been run over a range of values for the hyper-parameters, using the validation set for selection~\cite{bergstra2012random}. The hyper-parameters include the number of hidden-layer nodes, $H \in \{25, 50, 100\}$, the context window size, $s \in \{1, 3, 5\}$, and the embedding dimension, $d \in \{50, 100, 300, 500, 1000\}$. Two additional parameters, the learning and drop-out rates, were sampled from a uniform distribution in the range $[0.05, 0.1]$. The embedding and initial weight matrices were all sampled from the uniform distribution within range $[-1, 1]$. Early training stopping was set to $100$ epochs to mollify over-fitting, and the model that gave the best performance on the validation set was retained. The accuracy is reported in terms of micro-average F$_1$ score computed using the CoNLL score function~\cite{Nadeau:07}.



\section{Experimental results}
\label{sec:experiment-results}

\section{Experimental Results}
\label{sec:unsup_experiment-results}

In this section, we present experiments for three scenarios:
(a) non-inductive anomaly detection,
(b) inductive anomaly detection, and
(c) image denoising.

%\vspace{-0.3 cm}
%%%
\subsection{Non-inductive anomaly detection results}

We present results on the three datasets described in Section \ref{sec:experiment-setup}.


%%%
\subsubsection{{(1) {\tt restaurant} dataset}}
We work with the {\tt restaurant} video activity detection dataset~\cite{xiong2011direct},
and consider the problem of inferring the background of videos via removal of (anomalous) foreground pixels.
Estimating the background in videos is important for tasks such as anomalous activity detection.
It is however difficult because of the variability of the background (e.g. due to lighting conditions) and the presence of foreground objects such as moving objects and people.

For this experiment, we only compare the RPCA and RCAE methods, owing to a lack of ground truth labels.

\textbf{Parameter settings}.
For RPCA, rank $K$ = 64 is used.

Per the success of the Batch Normalization architecture~\cite{ioffe2015batch} and Exponential Linear Units~\cite{clevert2015fast}, we have found that convolutional+batch-normalization+elu layers provide a better representation of convolutional filters.
Hence, in this experiment the RCAE adopts four layers of (conv-batch-normalization-elu) in the encoder part and four layers of  (conv-batch-normalization-elu) in the decoder portion of the network.
RCAE network parameters such as (number of filter, filter size, strides) are chosen to be (16,3,1) for first and second layers and (32,3,1) for third and fourth layers of both encoder and decoder layers.

\begin{figure}[!t]
	\centering
	\subfigure[RCAE.]{\includegraphics[scale=0.325]{images/Restaurant/CAE_worst.jpg}}
	\subfigure[RPCA.]{\includegraphics[scale=0.325]{images/Restaurant/RPCA_worst.jpg}}
	\caption{Top anomalous images containing original image (people walking in the lobby) decomposed into background (lobby) and foreground (people), {\tt restaurant} dataset.}
	\label{fig:results-restaurant}
\end{figure}

%\vspace{-0.3 cm}
\textbf{Results}.
While there are no ground truth anomalies in this dataset, a qualitative analysis reveals RCAE to outperforms its counterparts in capturing the foreground objects.
Figure~\ref{fig:results-restaurant} compares the top 6 most anomalous images for RCAE and RPCA.
We see that the most anomalous images contain high foregound activity (which are recognised as anomalous).
Visually, we see that the background reconstruction produced by RPCA contains a few blemishes in some cases, while for RCAE the backgrounds are smooth.


%%%
\subsubsection{{(2) {\tt usps} dataset}}
From the {\tt usps} handwritten digit dataset,
we create a dataset
with a mixture of 220 images of \lq1\rq s, and 11 images of \lq7\rq, as in~\cite{xu2010robust}.
Intuitively, the latter images are treated as being anomalous, as the corresponding images have different characteristics to the majority of the training data. Each image is flattened as a row vector, yielding a 231 $\times$ 256 training matrix.

\textbf{Parameter settings}.
For SVD and RPCA methods, rank $K = 64$ is used.
For AE, the inputs are flattened images as a column vector of size 256,
and the hidden layer is a column vector of size  64 (matching the rank $K$).

For DRMF, we follow the settings of~\cite{xu2010robust}.
For RKPCA, we used a Gaussian kernel with bandwidth $0.01$, a cost parameter $C = 1$, and requested $60\%$ of the KPCA spectrum (which roughly selects 64 principal components).

For RCAE, we set two layers of convolution layers with the filter number to be 32, filter size to be 3$\times$3, with number of strides as $1$ and  rectified linear unit (ReLU) as activation with max-pooling layer of dimension 2$\times$2.

\textbf{Results}.
From Table~\ref{tbl:anomaly-results-summary}, we see that it is a near certainty for all \lq7\rq\, are accurately identified as outliers.
Figure~\ref{fig:usps-anomalies} shows the top anomalous images for RCAE, where indeed the \lq7\rq's are correctly placed at the top of the list.
By contrast, for RPCA there are also some \lq1\rq's placed at the top.

\begin{figure}[!t]
	\centering
	\subfigure[RCAE.]{\includegraphics[scale=0.9]{usps-anomalies-crop}}
	\quad
	\subfigure[RPCA.]{\includegraphics[scale=0.29]{rpca_usps}}
	\caption{Top anomalous images, {\tt usps} dataset.}
	\label{fig:usps-anomalies}
	%\vspace{-\baselineskip}
\end{figure}

\begin{table*}[!t]
	\centering
	\renewcommand{\arraystretch}{1.25}
	\resizebox{0.99\linewidth}{!}{
	\subfigure[{\tt usps}]{
		\begin{tabular}{lccc}
			\toprule
			\toprule
			\textbf{Methods} & \textbf{AUPRC} & \textbf{AUROC} & \textbf{P@10} \\
			\midrule
			RCAE  & \cellcolor{gray!25}{0.9614 $\pm$ 0.0025}&\cellcolor{gray!25}{0.9988$\pm$ 0.0243}&\cellcolor{gray!25}{0.9108 $\pm$ 0.0113} \\
			\midrule
			CAE & 0.7003 $\pm$ 0.0105 & 0.9712 $\pm$ 0.0002 & 0.8730 $\pm$ 0.0023\\
			AE  & 0.8533 $\pm$ 0.0023 & 0.9927 $\pm$ 0.0022 & 0.8108 $\pm$ 0.0003 \\
			\midrule
			RKPCA & 0.5340 $\pm$ 0.0262 & 0.9717 $\pm$ 0.0024 & 0.5250 $\pm$ 0.0307 \\
			DRMF  & 0.7737 $\pm$ 0.0351 & 0.9928 $\pm$ 0.0027 & 0.7150 $\pm$ 0.0342 \\
			RPCA  & 0.7893 $\pm$ 0.0195 & 0.9942 $\pm$ 0.0012 & 0.7250 $\pm$ 0.0323\\
			SVD   & 0.6091 $\pm$ 0.1263 & 0.9800 $\pm$ 0.0105 & 0.5600 $\pm$ 0.0249 \\
			\bottomrule
	\end{tabular}}%
	\quad
	\subfigure[{\tt cifar-10}]{
		\begin{tabular}{ccc}
			\toprule
			\toprule
			\textbf{AUPRC} & \textbf{AUROC} & \textbf{P@10} \\
			\midrule
			\cellcolor{gray!25}{0.9934 $\pm$ 0.0003}&\cellcolor{gray!25}{0.6255 $\pm$ 0.0055} &\cellcolor{gray!25}{0.8716 $\pm$ 0.0005} \\
			\midrule
			0.9011 $\pm$ 0.0000 & 0.6191 $\pm$ 0.0000 & 0.0000 $\pm$ 0.0000 \\
			0.9341 $\pm$ 0.0029 & 0.5260 $\pm$ 0.0003 & 0.2000 $\pm$ 0.0003 \\
			\midrule
			0.0557 $\pm$ 0.0037 & 0.5026 $\pm$ 0.0123 & 0.0550 $\pm$ 0.0185 \\
			0.0034 $\pm$ 0.0000 & 0.4847 $\pm$ 0.0000 & 0.0000 $\pm$ 0.0000 \\
			0.0036 $\pm$ 0.0000 & 0.5211 $\pm$ 0.0000 & 0.0000 $\pm$ 0.0000 \\
			0.0024 $\pm$ 0.0000 & 0.5299 $\pm$ 0.0000 & 0.0000 $\pm$ 0.0000 \\
			\bottomrule
	\end{tabular}}%
	}

	\caption{Comparison between the baseline (bottom four rows) and state-of-the-art systems (top three rows). Results are the mean and standard error of performance metrics over 20 random training set draws. Highlighted cells indicate best performer.}
	\label{tbl:anomaly-results-summary}
	%\vspace{-0.2in}
\end{table*}

\subsubsection{{(3) {\tt cifar-10} dataset}}
We create a dataset with anomalies
by combining 5000 random images of dogs and 50 images of cats, as illustrated in Figure~\ref{fig:results-cifar}.
In this scenario the cats are anomalies, and the goal is to detect all the cats in an unsupervised manner.

\textbf{Parameter settings}.
For SVD and RPCA methods, rank $K = 64$ is used.
We trained a three-hidden-layer autoencoder (AE) (1024-256-64-256-1024 neurons).
The middle hidden layer size is set to be same as rank $K = 64$, %(same as number of components used in SVD and RPCA),
and the model is trained using Adam~\cite{kingma2014adam}.
The decoding layer uses sigmoid function in order to capture the nonlinearity characteristics from  latent representations produced by the hidden layer.
Finally, we obtain the feature vector for each image by obtaining the latent representation from the hidden layer.

For RKPCA, we used a Gaussian kernel with bandwidth $5 \cdot 10^{-8}$, a cost parameter $C = 0.1$, and requested $55\%$ of the KPCA spectrum (which roughly selects 64 principal components).
The RKPCA runtime was prohibitive on the full sample (see Sec \ref{sec:runtime}), so we resorted to a subsample of 1000 dogs and 50 cats.

The RCAE architecture in this experiment is same as for {\tt restaurant}, containing
four layers of  (conv-batch-normalization-elu) in the encoder part
and four layers of  (conv-batch-normalization-elu) in the decoder
portion of the network. RCAE network parameters such as (number of filter, filter size, strides) are chosen to be (16,3,1)  for first and second layers and (32,3,1) for third and fourth layers of both encoder and decoder.

\begin{figure}[!t]
	\centering
	\subfigure[RCAE.]{\includegraphics[scale=0.95]{images/CIFAR_10/CAE_worst_inductive_crop}}
	\subfigure[RPCA.]{\includegraphics[scale=0.95]{images/CIFAR_10/RPCA_worst_inductive_crop}}
	\caption{Top anomalous images, %along with background and foreground,
	{\tt cifar-10} dataset.}
	\label{fig:results-cifar}
\end{figure}

\textbf{Results}.
From Table \ref{tbl:anomaly-results-summary},
RCAE clearly outperforms all existing state-of-the art methods in anomaly detection.
Note that basic CAE, with no robustness (effectively $\lambda = \infty$), is also outperformed by our method, indicating that it is crucial to explicitly handle anomalies with the $\N$ term.

Figure~\ref{fig:results-cifar} illustrates the most anomalous images for our RCAE method, compared to RPCA.
Owing to the latter involving learning a linear subspace, the model is unable to effectively distinguish cats from dogs;
by contrast, RCAE can effectively determine the manifold characterising most dogs, and identifies cats to be anomalous with respect to this.


%%%%%%%%%
\subsection{Inductive anomaly detection results}

We conduct an experiment to assess the detection of \emph{inductive} anomalies.
Recall that this is a capability of our RCAE model, but not e.g. RPCA.
We consider the following setup:
we train our model on 5000 dog images, and then evaluate it on a test set comprising 500 dogs and 50 cat images.
As before, we wish all methods to accurately determine the cats to be anomalies.

\begin{figure}[!t]
	\centering
	\subfigure[RCAE.]{\includegraphics[scale=0.31]{images/CIFAR_10/caeInductiveAnomalies_crop.jpg}}
	\subfigure[CAE.]{\includegraphics[scale=0.95]{images/CIFAR_10/caeInductiveAnomalies_crop_new.jpg}}
	\caption{Top inductive anomalous images, {\tt cifar-10} dataset.}
	\label{fig:results-cifar-inductive}
\end{figure}


Table~\ref{tbl:inductive-anomaly-results} summarises the detection performance for all the methods on this inductive task.
The lower values compared to Table~\ref{tbl:anomaly-results-summary} are indicative that the problem here is more challenging than anomaly detection on a single dataset;
nonetheless, we see that our RCAE method manages to convincingly outperform both the SVD and AE baselines.
This is confirmed qualitatively in Figure~\ref{fig:results-cifar-inductive}, where we see that RCAE correctly identifies many cats in the test set as anomalous, while the basic
%AE
CAE
method suffers.

\begin{table}[!t]
	\centering
	\renewcommand{\arraystretch}{1.25}
	\caption{Inductive anomaly detection results on {\tt cifar-10}. Note that RPCA and DRMF are inapplicable here. Highlighted cells indicate best performer.}
	\resizebox{0.99\linewidth}{!}{
	\begin{tabular}{@{}lccccc@{}}
		\toprule
		\toprule
		& \textbf{SVD} & \textbf{RKPCA} & \textbf{AE} & \textbf{CAE} & \textbf{RCAE} \\
		\toprule
		\bf{AUPRC} & 0.1752 $\pm$ 0.0051 & 0.1006 $\pm$ 0.0045 & 0.6200 $\pm$ 0.0005 & 0.6423 $\pm$ 0.0005 & \cellcolor{gray!25}{0.6908 $\pm$ 0.0001} \\
		\bf{AUROC} & 0.4997 $\pm$ 0.0066 & 0.4988 $\pm$ 0.0125 & 0.5007 $\pm$ 0.0010 & 0.4708 $\pm$ 0.0003 & \cellcolor{gray!25}{0.5576 $\pm$ 0.0005} \\
		\bf{P@10}  & 0.2150 $\pm$ 0.0310 & 0.0900 $\pm$ 0.0228 & 0.1086 $\pm$ 0.0001 & 0.2908 $\pm$ 0.0001 & \cellcolor{gray!25}{0.5986 $\pm$ 0.0001} \\
		\bottomrule
	\end{tabular}
	}
	\label{tbl:inductive-anomaly-results}
\end{table}


%%%
\subsection{Image denoising results}

Finally, we test the ability of the model to de-noise images, which is a form of anomaly detection on individual pixels (or more generally, features).
In this experiment, we train all models on a set of 5000 images of dogs from {\tt cifar-10}.
For each image, we then add salt-and-pepper noise at a rate of 10\%.
Our goal is to recover the original image as accurately as possible.

Figure~\ref{fig:results-cifar-injection} illustrates that the most anomalous images in the presence of noise
contain images of the variations of dog class images (e.g. containing person's face).
Further, Figure \ref{fig:denoising-results} illustrates for various methods the mean square error between the reconstructed and original images.
RCAE effectively suppresses the noise as evident from the low error.
The improvement over raw CAE is modest, but suggests that there is benefit to explicitly accounting for noise.

%\vspace{-0.3 cm}
\begin{figure}[!t]
	\centering
	\subfigure[RCAE.]{\includegraphics[scale=0.95]{images/CIFAR_10/CAE_worst_crop.jpg}}
	\subfigure[RPCA.]{\includegraphics[scale=0.95]{images/CIFAR_10/RPCA_worst_crop.jpg}}
	\caption{Top anomalous images in original form (first row), noisy form (second row),
	image denoising task on {\tt cifar-10}.}
	\label{fig:results-cifar-injection}
\end{figure}

\begin{figure}[!t]
	\centering
	\includegraphics[scale=0.5]{images/cifar_10_mse.pdf}
	\caption{Illustration of the mean square error boxplots obtained for various models on image denoising task, {\tt cifar-10} dataset.
		In this setting, RCAE suppresses the noise and detects the background and foreground images effectively.}
	\label{fig:denoising-results}
	%\vspace{-\baselineskip}
\end{figure}


%%%
\subsection{Comparison of training times}
\label{sec:runtime}

We remark finally that our RCAE method is comparable in training efficiency to existing methods.
For example, on the small-scale {\tt restaurant} dataset, it takes 1 minute to train RPCA,
and 8.5 minutes to train RKPCA,
compared with 10 minutes for our RCAE method.
The ability to leverage recent advances in deep learning as part of our optimisation (e.g.\ training models on a GPU) is we believe a salient feature of our approach.

We note that while the RKPCA method is fast to train on smaller datasets, on larger datasets it suffers from the $O(n^2)$ complexity of kernel methods;
for example, it takes over an hour to train on the {\tt cifar-10} dataset.
It is plausible that one could leverage recent advances in fast approximations of kernel methods~\cite{Lopez-Paz:2014}, and studying these would be of interest in future work.
Note that the issue of using a fixed kernel function would remain, however.


\section{Conclusion}
\label{sec:conclusion}
% !TEX root=../main.tex
%By modeling images as group anomalies
\section{Conclusion}
\label{sec:conclusion}
This paper has investigated the effectiveness of recurrent neural architectures, namely the Elman and Jordan networks and the bidirectional LSTM-CRF, for drug name recognition. The most appealing feature of these architectures is their ability to provide end-to-end recognition straight from text, sparing effort from laborious feature construction. To the best of our knowledge, ours is the first paper to explore RNNs for entity recognition from pharmacological text. The experimental results over the SemEval-2013 Task 9.1 benchmarks look promising, with the bidirectional LSTM-CRF ranking closely to the state of the art. A potential way to  further improve its performance would be to initialize its training with unsupervised word embeddings such as Word2Vec~\cite{Mikolov:13} and GloVe~\cite{Pennington:14}. This approach has proved effective in many other domains and still dispenses with expert annotation effort; we plan this exploration for the near future.






\bibliographystyle{splncs03}
\bibliography{MyBibFile,outlier}

\end{document}
