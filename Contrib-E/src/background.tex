% !TEX root=../main.tex
\section{Related Work} \label{DGM:RelatedWork}
GAD is an emerging area  of research where most state-of-the-art techniques have been more recently developed.  While group anomalies are briefly discussed in anomaly detection surveys such as Chandola et al. \cite{Chandola} and Austin \cite{Hodge},  
  Yu et al. \cite{SurveySocialMedia} elaborates on GAD techniques where group memberships are not previously known.  
   {  Recently Toth and Chawla \cite{MySurvey} provide a comprehensive overview of GAD methods as well as a detailed  description of detecting temporal changes in groups over time. 
}
We focus on group anomalies in image applications where group memberships are known a priori. 

Previous studies on image anomaly detection % involving image   applications 
can be understood in terms of group anomalies. 
%Anomaly detection has been  previously studied in a variety of image classification applications. 
Quellec et al.  \cite{mammo} examine mammographic images  where point-based group anomalies represent potentially cancerous regions. Perera and Patel \cite{chairs} learn features from a collection of images containing regular chair objects and detect point-based group anomalies where chairs have abnormal shapes, colors and other irregular characteristics. On the other hand, Xiong et al. \cite{FGM} detect distribution-based group anomalies that are stitched images from scene categories (inside city, mountain or coast). %At a pixel level,   Xiong et al. \cite{MGM} apply GAD methods to detect anomalous galaxy clusters with irregular proportions of RGB pixels. 
We emphasise detecting distribution-based group anomalies rather than point-based anomalies in our subsequent  image applications.

The discovery of group anomalies is of interest to a number of diverse domains.   
  Muandet et al.	\cite{OCSMM} investigate GAD for physical phenomena in high energy particle physics where Higgs bosons are observed as slight excesses in a collection of collision events rather than individual  events. Xiong et al. \cite{FGM} analyse fluid dynamics  where a group anomaly represents unusual vorticity and turbulence in  fluid motion.  % In topic modeling,  Soleimani and Miller \cite{ATD} characterise documents by topics and anomalous clusters of documents are discovered by their irregular  topic mixtures. 
   By incorporating additional information from pairwise connection data, Yu et al. \cite{GLAD} find potentially irregular communities of co-authors in various research communities.
Thus there are many GAD application other than image anomaly detection.
 

A  related discipline to image anomaly detection is video anomaly detection where many DGMs are applied.  
 Sultani  et al.  \cite{survideos1} detect real-world anomalies such as burglary, fighting, vandalism and so on from  CCTV footage using deep learning methods.  %Many techniques involving deep learning architectures have been applied to detect temporal changes in a video surveillance application.    
% In a review, Kiran et al. \cite{survideos2} compare DGMs with different  convolution architectures for  video anomaly detection applications. 
Recent work ~\cite{schlegl2017unsupervised,xu2018unsupervised,an2015variational} illustrate the effectiveness of generative models for high-dimensional anomaly detection. Although, there are existing works that apply DGMs in image-related applications, we leverage  autoencoders for DGMs  to detect group anomalies in a variety of image experiments. 
 

