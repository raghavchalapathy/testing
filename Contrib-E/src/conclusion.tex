% !TEX root=../main.tex
%By modeling images as group anomalies
\section{Conclusion} 
\label{sec:conclusion} 
Group anomaly detection (GAD) is a challenging area of research especially when dealing with complex group distributions such as image data. In this chapter, we clearly formulate deep generative models (DGMs)  for  detecting group anomalies.   %Although DGMs have been previously applied in various image applications, we formulate DGMs such as VAE and AAE for solving the GAD problem using certain characterisation and group reference functions.  
 DGMs outperform state-of-the-art GAD techniques in many experiments involving both synthetic and real-world image datasets however DGMs require a large number of group observations for model training. To the best of our knowledge, we are the first to formulate and apply DGMs to the problem of detecting group anomalies. A future direction for research involves using recurrent neural networks  to detect  temporal changes in a group of time series. 
 



