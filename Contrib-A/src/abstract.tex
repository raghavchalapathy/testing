% !TEX root=../main.tex


Unlike conventional anomaly detection research that focuses on point anomalies, our goal is to detect anomalous collections of individual data points. In particular, we perform group anomaly detection (GAD) with an emphasis on irregular group distributions (e.g. irregular mixtures of image pixels). GAD is an important task in detecting unusual and anomalous phenomena in  real-world applications such as high energy particle physics, social media and medical imaging. In this paper, we take a generative approach by proposing deep generative models: Adversarial autoencoder (AAE) and variational autoencoder (VAE) for group anomaly detection. Both AAE and VAE detect group anomalies using point-wise input data where group memberships are known a priori. We conduct extensive experiments to evaluate our models on real world datasets. The empirical results demonstrate that our approach is effective and robust in detecting group anomalies.



%\keywords{group anomaly detection, adversarial, variational , auto-encoders}
