% !TEX root=../main.tex
\section{Related Work}
\label{supervisedDAD:RelatedWork}
Most of the research on drug name recognition to date has focussed on domain-dependent aspects and specialized text features. The benefit of leveraging such tailored features was made evident by the results from the SemEval-2013 Task 9.1 (Recognition and classification of pharmacological substances, known as DNR task) challenge. The system that ranked first, WBI-NER~\cite{huber2013wbi}, adopted very specialized features derived from an improved version of the ChemSpot tool~\cite{rocktaschel2012chemspot}, a collection of drug dictionaries and ontologies. Similarly, many other recent approaches~\cite{abacha2015text,liu2015feature,segura2015exploring} have been based on various combinations of general and domain-specific features. In the broader field of machine learning, the recent years have witnessed a rapid proliferation of deep neural networks, with unprecedented results in tasks as diverse as visual, speech and named-entity recognition \cite{hinton2012deep,krizhevsky2012imagenet,lample2016neural}. One of the main advantages of neural networks is that they can learn the feature representations automatically from the data, thus avoiding the laborious feature engineering stage~\cite{mesnil2015using,lample2016neural}. Given these promising results, the main goal of this paper is to provide the first performance investigation of popular RNNs such as the Elman and Jordan networks and the bidirectional LSTM-CRF over DNR tasks.
